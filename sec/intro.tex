\section{Introduction to STE}

\subsection{Original STE}

\textit{Symbolic trajectory evaluation}~\cite{seger1995} (STE) is a high-performance model checking technique based on \textit{symbolic simulation} extended with a temporal \textit{next-time} operator to describe circuit behaviour over time. In its simplest form, STE tests the validity of an \textit{assertion} of the form $A \Rightarrow C$, where both the \textit{antecedent} $A$ and \textit{consequent} $C$ are formulas in the following logic:

\begin{equation*}
f ::= p \: | \: f \wedge f \: | \: P \rightarrow f \: | \: \Next f
\end{equation*}

\noindent Here, $p$ is a simple predicate over ``values'' in a circuit and $P$ is a Boolean propositional formula, and the operators $\wedge$, $\rightarrow$ and $\Next$ are conjunction, domain restriction and the next-time operator, respectively.

If the circuit contains Boolean signals, $p$ is typically drawn from the following two predicates: $n \Is 1$ and $n \Is 0$, where $n$ ranges over the signals (or nodes) in a circuit. For example, suppose we have a unit-delayed two-input AND-gate, then it is reasonable to assume that the assertion $(\mathit{in_{1}} \Is 1 \wedge \mathit{in_{2}} \Is 1) \Rightarrow \Next (\mathit{out} \Is 1)$ is true. Indeed, STE efficiently validates such statements for us.

While the truth semantics of an assertion in STE is defined as the satisfaction of its ``defining'' trajectory (bounded sequence of states) relative to a model structure of the circuit, what the STE algorithm computes is exactly the solution of a data-flow equation~\cite{chou1999} in the classic format~\cite{muchnick1997}. \dots

% translate an STE assertion into a linear, directed graph \dots introducing as many states as the assertion is deep \dots labelling each state with the antecedent and consequent at that depth. \dots

% We adopt this reimagined STE as data flow analysis~\cite{chou1999}, in which trajectory assertions can have arbitrary state-transition graphs. The following sections give a short introduction to its mathematical foundation.

\subsection{Lattice-theoretic STE}

\subsubsection{Circuit Models}

Consider an arbitrary, but fixed, digital circuit $M$ operating in discrete time. A \textit{configuration} of $M$, denoted by $C$, is non-empty and finite set that represents a snapshot of $M$ at a discrete point in time. If the circuit $M$ has $m$ boolean signals, then its set of configurations is typically represented as a sequence $\mathbb{B}^{m}$, where $\mathbb{B} = \{ 0,1 \}$ is the set of boolean values.

A simple conceptual model of $M$ is a \textit{transition relation}, $M_{R} \subseteq C \times C$, where $(c,c') \in M_{R}$ means that $M$ can move from $c$ to $c'$ in one step\footnotemark. The power set of $C$, denoted by $\mathcal{P}(C)$, can be viewed as a the set of \textit{predicates} on configurations, where $\cap$, $\cup$, and $\subseteq$ correspond to conjunction, disjunction and implication, respectively. Furthermore, for any $Q \subseteq \mathcal{P}(C)$, we denote by $\cap Q$ and $\cup Q$ the intersection and union of all members of $Q$.

\footnotetext{\textcolor{red}{Mention how this affects circuits with zero-delays?}}

$M_{R}$ induces a \textit{predicate transformer} $M_{F} \in \mathcal{P}(C) \rightarrow \mathcal{P}(C)$ using the relational image operation:

\begin{equation*}
M_{F}(p) = \{ c' \in C \mid \exists c \in p : (c,c') \in M_{R} \}
\end{equation*}

\noindent It is intuitively obvious that if $M$ is in one of the configurations in $p \in \mathcal{P}(C)$, then in one time step it must be in one of the configurations in $M_{F}(p)$, or in other words, $(c,c') \in M_{R} \iff c' \in M_{F}(\{ c \})$. We adopt this functional model of $M$ and drop its subscript.

Since the value of input signals is controlled by the external environment, a circuits model itself does not impose any constraint on them. For such non-input signals, the model is a montonic function where imposed constraints are determined by the circuit topology and functionality. For instance, if assume the circuit is a two-input AND gate followed by an inverter, its model have the type $M \in \mathcal{P}(\mathbb{B}^{4}) \rightarrow \mathcal{P}(\mathbb{B}^{4})$ and can be defined as:

\begin{equation*}
M(p) = \{ \langle b_{1}, b_{2}, i_{1} \wedge i_{2}, \neg s \rangle \in \mathbb{B}^{4} \mid \langle i_{1}, i_{2}, s, o \rangle \in \cap p \}
\end{equation*}

\noindent Here $i_{1}$ and $i_{2}$ refer to the two inputs of the AND gate, $s$ to the internal inverter, and $o$ is the ignored output; $b_{1}$ and $b_{2}$ are unconstrained in the new configuration.

Manipulating subsets of $\mathbb{B}^{m}$ is however impractical for even moderately large $m$, which leads us to one of the key insights of STE. Namely, instead of manipulating subsets of $\mathbb{B}^{m}$ directly, one can use sequences of ternary values $\mathbb{T} = \mathbb{B} \cup \{ X \} $ to approximate them, whose sizes are only linear in $m$. Here the $1$ and $0$ from $\mathbb{B}$ denotes specific, defined values whereas $X$ denotes an ``unknown'' value that could be either $1$ or $0$. This intuition induces a partial order $\sqsubseteq$ on $\mathbb{T}$, where $0 \sqsubseteq X$ and $1 \sqsubseteq X$\footnotemark. For any $m \in \mathbb{N}$, this ordering on $\mathbb{T}$ is lifted component-wise to $\mathbb{T}^{m}$.

\footnotetext{We use the reverse ordering of what is originally used in STE.}

\subsubsection{Ternary lattices}

Let $(\hat P,\sqsubseteq)$ be a finite, complete lattice of \textit{abstract predicates} in which the meet $\sqcap$ and join $\sqcup$ of any subset $Q \subseteq \hat P$ exists. Similar to the previous set operations for power sets, $\sqcap$, $\sqcup$ and $\sqsubseteq$ correspond to conjunction, disjunction and implication for abstract predicates, respectively. Furthermore, for any $Q \subseteq \hat P$, we denote by $\sqcap Q$ and $\sqcup Q$ the meet and join of all members of $Q$. 

Note that $\mathbb{T}^{m}$ does not quite form a complete lattice because it lacks a bottom: both $0 \sqsubseteq X$ and $1 \sqsubseteq X$ but $0$ and $1$ are equally defined. A special bottom element $\bot$ is therefore introduced, such that $\bot \sqsubseteq t$ and $\bot \neq t$ for all $t \in \mathbb{T}^{m}$. The extended $\mathbb{T}_{\bot}^{m} = \mathbb{T}^{m} \cup \{ \bot \}$ then becomes a complete lattice. We denote its top element $\langle X, \dots, X \rangle$ by $\top$.

Let there also be Galois connection $\ll \, \in \, \subseteq \mathcal{P}(C) \times \hat P$, relating concrete and abstract predicates such that for all $Q \subseteq \mathcal{P}(C)$ and $\hat Q \subseteq \hat P$:

\begin{equation*}
Q \ll \hat Q \iff \cup Q \ll \sqcap \hat Q
\end{equation*}

\noindent where $Q \ll \hat Q \iff \forall p \in Q : \forall \hat p \in \hat Q : p \ll \hat p$. Intuitively, $p \ll \hat p$ states that $p$ can be ``approximated'' as $\hat p$, and we note that $\ll$ is an extension of the partial orders of $\mathcal{P}(C)$ and $\hat P$ to an ordering between $\mathcal{P}(C)$ and $\hat P$.

It is sometimes convenient to define a Galois connection in terms of an \textit{abstraction} $\alpha \in \mathcal{P}(C) \rightarrow \hat P$ and a \textit{concretisation} $\gamma \in \hat P \rightarrow \mathcal{P}(C)$ function, from which one can derive $\ll$ as follows: $p \ll \hat p \iff p \subseteq \gamma(\hat p)$ or $p \ll \hat p \iff \alpha (p) \sqsubseteq \hat p$. For example, a Galois connection from $\mathcal{P}(\mathbb{B}^{m})$ to $\mathbb{T}_{\bot}^{m}$ for any $m \in \mathbb{N}$ can be defined in a natural way by specifying its concretisation function $\Gamma \in \mathbb{T}_{\bot}^{m} \rightarrow \mathcal{P}(\mathbb{B}^{m})$:

\begin{align*}
\Gamma ( \langle t_{0},\dots,t_{m-1} \rangle ) &= \{ \langle b_{0},\dots,b_{m-1} \rangle \in \mathbb{B}^{m} \mid \forall i < m : t_{i} \neq X \Rightarrow b_{i} = t_{i} \} \\
\Gamma ( \bot ) &= \emptyset
\end{align*}

\noindent Listing each boolean sequence approximated by the given ternary sequence. For example, $\langle 1,X,0 \rangle$ abstracts both the boolean sequence $\langle 1,1,0 \rangle$ and $\langle 1,0,0 \rangle$ since they all agree on their first and third element.

An \textit{abstract predicate transformer} $\hat M \in \hat P \rightarrow \hat P$ is an \textit{abstract interpretation}~\cite{cousot1996} of $M \in \mathcal{P}(C) \rightarrow \mathcal{P}(C)$ iff it preserves $\bot$, $\hat M(\bot) = \bot$; is monotonic, $\hat p \sqsubseteq \hat q \Rightarrow \hat M (\hat p) \sqsubseteq \hat M (\hat q)$ for all $\hat p, \hat q \in \hat P$; and $\ll$ is a \textit{simulation relation} from $\mathcal{P}(C)$ to $\hat P$, $p \ll \hat p \Rightarrow M (p) \ll \hat M (\hat p)$ for all $p \in \mathcal{P}(C)$ and $\hat p \in \hat P$.

\textcolor{red}{Example of how $\hat M$ and $M$ relate in practice.}

\subsubsection{Assertions and satisfaction} A \textit{trajectory assertion} for $\hat M$ is a quintuple $\hat A = (S,s_{0},R,\pi_{a},\pi_{c})$, where $S$ is a finite set of \textit{states}, $s_{0} \in S$ is an \textit{initial state}, $R \subseteq S \times S$ is a \textit{transition relation}, $\pi_{a} \in S \rightarrow \hat P$ and $\pi_{c} \in S \rightarrow \hat P$ label each state $s$ with an \textit{antecedent} $\pi_{a}(s)$ and a \textit{consequent} $\pi_{c}(s)$. Furthermore, we assume that $(s,s_{0}) \notin S$ for all $s \in S$ without any loss of generality.

For all $\Phi \in S \rightarrow \hat P$ and $s \in S$, define $F \in S \rightarrow (\hat P \rightarrow \hat P)$ and $\mathcal{F} \in (S \rightarrow \hat P) \rightarrow (S \rightarrow \hat P)$ as follows:

\begin{align}
F(s)(\hat p) &= \hat M(\pi_{a}(s) \sqcap \hat p) \\
\mathcal{F}(\Phi)(s) &= \Stmt{if } (s = s_{0}) \Stmt{ then } \top \Stmt{ else } \sqcup \{ F(s')(\Phi(s')) \mid (s',s) \in R \}
\end{align}

\noindent $F$ preserves $\bot$ and both $F$ and $\mathcal{F}$ are monotonic, where two $\Phi, \Phi' \in S \rightarrow \hat P$ are ordered as $\Phi \sqsubseteq \Phi' \iff \forall s \in S : \Phi(s) \sqsubseteq \Phi'(s)$. Let $\Phi_{*} \in S \rightarrow \hat P$ be the least fixpoint of the equation $\Phi = \mathcal{F}(\Phi)$~\cite{davey2002}. Since both $S$ and $\hat P$ are finite, $\Phi_{*}$ is given by $\lim \, \Phi_{n}(s)$ where $\Phi_{n}$ is defined as follows:

\begin{equation}
\Phi_{n} = \Stmt{if } (n = 0) \Stmt{ then } (\lambda s \in S : \bot) \Stmt{ else } \mathcal{F}(\Phi_{n-1})
\end{equation}

We say that the abstract circuit $\hat M$ \textit{satisfies} a lattice-based, abstract trajectory assertion $\hat A$, denoted by $\hat M \models \hat A$, iff:

\begin{equation}
\forall s \in S : \Phi_{*}(s) \sqcap \pi_{\alpha}(s) \sqsubseteq \pi_{c}(s)
\end{equation}

\noindent $\hat M \models \hat A$ implies that a concretisation of $\hat A$ can also be satisfied by the original, set-based model $M$ \cite{chou1999}.
