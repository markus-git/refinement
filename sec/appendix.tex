\section{Proofs (WIP)}

\subsection{$\hat M \enc \hat N \iff \hat M \leq \hat N$}

We prove the two directions of $\hat M \enc \hat N \iff \hat M \leq \hat N$ separately.

\subsubsection{$\hat M \enc \hat N \implies \hat M \leq \hat N$:} For any $\delta \in \hat I^{+}$, let $\tau \in \hat P^{+}$ and $v \in \hat Q^{+}$ be the induced trajectories $\Driv{\hat M}{\delta}$ and $\Driv{\hat N}{\delta}$, respectively. We first prove that $\tau[k] \enc v[k]$ for each $\delta \in \hat I^{+}$ and $k \in \mathbb{N} : k < | \tau | = | v | = | \delta + 1 |$ by induction on $k$. The base case, $\tau[0] = (\top \in \hat P) \enc (\top \in \hat Q) = v[0]$, follows immediately from the definition of $\hat M \enc \hat N$. For the inductive step, assume that $\tau[k] \enc v[k]$. That also $\tau[k+1] = \hat M(\delta[k] \sqcap \tau[k]) \enc \hat N(\delta[k] \sqcap v[k]) = v[k+1]$ follows from the second property of $\tau[k] \enc v[k]$ since $\delta[k] \in \hat I$. Finally, for any $\delta \in \hat I^{+}$, that $\out{\tau[k]} \sqsubseteq \out{v[k]}$ for each $k$ follows from the first property of $\tau[k] \enc v[k]$.

% Consider an arbitrary $\delta \in \hat I^{+}$ and the two trajectories it induces, $\tau \in \Driv{\hat M}{\delta}$ and $v \in \Driv{\hat N}{\delta}$. For index $0$, prop. 2 of $\hat M \enc \hat N$ and $\delta[0] \in \hat I$ implies that $\tau[0] = \hat M(\delta[0]) = \hat M(\delta[0] \sqcap \top) \enc \hat N(\delta[0] \sqcap \top) \hat N(\delta[0]) = v[0]$, from which prop. 1 states that $\out{\tau[0]} \sqsubseteq \out{v[0]}$. For index $1$, since $\delta[1] \in \hat I$, prop. 2 of $\hat M(\tau[0]) \enc \hat N(v[0])$ implies that $\tau[1] = \hat M(\delta[1] \sqcap \tau[0]) \sqsubseteq \hat N(\delta[1] \sqcap v[0]) = v[1]$, from which prop. 1 states that $\out{\tau[1]} \sqsubseteq \out{v[1]}$. Etc. for higher indices.

\subsubsection{$\hat M \enc \hat N \Leftarrow \hat M \leq \hat N$:} Define $\enc \in \hat P \times \hat Q$ as follows:

\begin{equation*}
\bigcup_{\delta \in I^{+}} \{ (\Driv{\hat M}{\delta}[k], \Driv{\hat N}{\delta}[k]) \mid k \in \mathbb{N}, k < | \delta + 1 | \}
\end{equation*}

\noindent For any $\hat p \enc \hat q$, we thus know that $\hat p = \tau[k]$ and $\hat q = v[k]$ for some pair of trajectories, $\tau$ and $v$, induced by some common driver $\delta$. That $\out{\hat p} \enc \out{\hat q}$ then follows from how $\hat M \leq \hat N$ implies that $\out{\tau[k]} \sqsubseteq \out{v[k]}$ for each $k$. Further, $\delta$ followed by any $\hat \iota \in \hat I$ is also a valid driver in $\hat I^{+}$. Thus, $\hat M(\hat \iota \sqcap \hat p) \enc \hat N(\hat \iota \sqcap \hat q)$ must hold for all $\hat \iota \in \hat I$ as well. That $\hat M \enc \hat N$ then follows from how $\out{\top \in \hat P} \sqsubseteq \out{\top \in \hat Q}$ is obviously true and that $\hat M(\hat \iota \sqcap \top) \enc \hat N(\hat \iota \sqcap \top)$ for every $\hat \iota \in \hat I$ since $\langle \hat \iota \rangle \in \hat I^{+}$.

% $\delta^{\frown}\hat \iota \in \hat I^{+}$

% Prop. 1 of $\hat M \enc \hat N$, that $\top \enc \top$, is obviously true. For prop. 2, we note that every $\hat \iota_{0} \in \hat I$ also forms a valid 1-length driver $\langle \hat \iota_{0} \rangle \in \hat I^{+}$. Combined with $\hat M \leq \hat N$, we then have that $\out{\hat M(\hat \iota_{0} \sqcap \top)} = \out{\hat M(\hat \iota_{0})} \sqsubseteq \out{\hat N(\hat \iota_{0})} = \out{\hat N(\hat \iota_{0} \sqcap \top)}$ for all $\hat \iota_{0} \in \hat I$. The relation thus holds at least for the first step. But for any choice of $\hat \iota_{0}$, every following choice of $\hat \iota_{1} \in \hat I$ also forms a valid 2-length driver $\langle \hat \iota_{0}, \hat \iota_{1} \rangle \in \hat I^{+}$. We therefore have that $\out{\hat M(\hat \iota_{1} \sqcap \hat M(\hat \iota_{0}))} \sqsubseteq \out{\hat N(\hat \iota_{1} \sqcap \hat N(\hat \iota_{0}))}$ for all $\hat \iota_{0}, \hat \iota_{1} \in \hat I$ and the relation thus hold for a second step as well. Etc. for further steps. \textcolor{red}{Build this relation and show that it has the desired properties.}

% \textcolor{red}{Construct the simulation relation.}
