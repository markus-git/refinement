\section{Appendices}

\subsection{Theorem~\ref{thm:traj-refines}}

We first prove a few lemmas.

\begin{lemma} \label{lem:class-sub}
$[d] \cap \Antecedent(\rho) \neq \emptyset \implies [d] \subseteq \Antecedent(\rho)$
\end{lemma}

Since they intersect, there must exist $d' \in [d]$ such that $d' \in \Antecedent(\rho)$. By property \todo{name} of $\Antecedent(\rho)$, it must be that $[d'] = [d] \subseteq \Antecedent(\rho)$. $\qedbox$

\begin{lemma} \label{lem:traj-con}
$d \in \Consequent(\rho) \wedge \{ c \} \ll \{ d \} \implies c \in \gamma(\Consequent(\rho))$
\end{lemma}

By property \todo{name} of $\Consequent$, $d \in \Consequent(\rho) \implies [d] \subseteq \Consequent(\rho)$ which, by the monotonicity of $\gamma$, implies $\gamma([d]) \subseteq \gamma(\Consequent(\rho))$. By definition of $\{ c \} \ll \{ d \}$, we know $[c] \subseteq \gamma([d])$, thus $[c] \subseteq \gamma(\Consequent(\rho))$. $\qedbox$

\begin{lemma} \label{lem:traj-ant}
$c \in \gamma(\Antecedent(\rho)) \wedge \{ c \} \ll \{ d \} \implies d \in \Antecedent(\rho)$
\end{lemma}

By property \todo{name} of $\Antecedent$ and the definition of $\gamma(\Antecedent)$, $c \in \gamma(\Antecedent(\rho))$ states that $[c] \subseteq \gamma(\Antecedent(\rho))$. And thus $\alpha([c]) \subseteq \alpha(\gamma(\Antecedent(\rho)) \subseteq \Antecedent(\rho)$ by the monotonicity of $\alpha$. By definition $\{ c \} \ll \{ d \}$, we also have that $\alpha([c]) \subseteq [d]$. Since $\alpha([c]) \neq \emptyset$, it must be that $[d] \cap \Antecedent(\rho) \neq \emptyset$, and thus $[d] \subseteq \Antecedent(\rho)$ by lemma~\ref{lem:class-sub}. Then, by property \todo{name} of $\Antecedent$, we must have that $d \in \Antecedent(\rho)$. $\qedbox$
\\

%% Because $\forall x \in \alpha([c]) : [x] = [d]$, property \todo{name} of $\Antecedent$ and implies that $[d] \subseteq \Antecedent(\rho) \implies d \in \Antecedent(\rho)$. 

For the theorem, we are given $\tau \in \Traj(M)$ and $\rho \in \Runs(\gamma(A))$, such that $| \tau | = | \rho |$ and $\tau_{n} \in \gamma(\Antecedent(\rho_{n}))$ for all $n \in \mathbb{N} : n < | \tau |$. We must then show that $\tau_{n} \in \gamma(\Consequent(\rho_{n}))$. By the refinement assumption, there must exist a $\upsilon \in \Traj(N)$ such that $| \tau | = | \upsilon | = | \rho |$ and $\{ \tau_{n} \} \ll \{ \upsilon_{n} \}$. By lemma~\ref{lem:traj-ant}, we know $\upsilon_{n} \in \Antecedent(\rho_{n})$ and thus $\upsilon_{n} \in \Consequent(\rho_{n})$. Lemma~\ref{lem:traj-con} then states that $\tau_{n} \in \gamma(\Consequent(\rho_{n}))$. $\qedbox$

\subsection{Theorem~\ref{thm:traj-equal-set}}

%% \begin{lemma} \label{lem:name2}
%% $C \ll D \wedge C' \ll D' \implies (C \cup C') \, \ll (D \cup D')$
%% \end{lemma}

%% As both $\alpha(C) \subseteq D \subseteq (D \cup D')$ and $\alpha(C') \subseteq D' \subseteq (D \cup D')$, it follows from how $\alpha$ distributes over arbitrary union that $\alpha(C) \cup \alpha(C') = \alpha(C \cup C') \subseteq (D \cup D')$. \todo{\dots}

\begin{lemma} \label{lem:ll-sub}
$C \ll D \wedge D \subseteq D' \implies C \ll D'$
\end{lemma}

That $C$ is approximated by $D'$ follows immediately: $\alpha(C) \subseteq D \subseteq D'$. The first property of $\ll$ follows from the definition of subset, and the second by the monotonicity of $N$: $\alpha(M(C)) \subseteq N(D) \subseteq N(D')$. $\qedbox$
\\

We prove each direction of the theorem separately.

$(\Rightarrow)$ : If $C \ll D$, then by definition $\alpha([C]) \subseteq [D]$. As $\alpha$ distributes over arbitrary union, it follows that $\alpha([c]) \subseteq [D]$ for all $c \in C$. Furthermore, applying the assumption $M \Refines N$ to one-length trajectories starting in $c$, there must also exist $d \in \D$ such that $\{ c \} \ll \{ d \}$, or $\alpha([c]) \subseteq [d]$ by definition of $\ll$. Since $\alpha([c]) \neq \emptyset$, it must be that $[d] \cap [D] \neq \emptyset$. By lemma~\ref{lem:class-sub} then, we know $[d] \subseteq [D]$ and thus $d \in D$. This shows that the first property of $\ll$ is implied. For the second property, that $\ll$ is a simulation relation, consider all two-length trajectories from $C$. For any $c' \in M(C)$, there exists a $d' \in N(D)$ such that $\{ c' \} \ll \{ d' \}$, which implies that $\{ c' \} \ll N(D)$ by lemma~\ref{lem:ll-sub}. Combining all such orderings, we have the desired $M(C) \ll N(D)$. $\qedbox$

% i.e. $\alpha([C]) = \cup \{ \alpha([c]) \in \pow(\C) \mid c \in C \}$

$(\Leftarrow)$ : We show this claim by induction on the length of $\tau$. For the base case, when $| \tau | = 1$, we are given $\tau = \langle \tau_{1} \rangle$ where $\tau_{1}$ is unconstrained, i.e. we only know that $\tau_{1} \in \C$. But a Galois connection always relates the most general states of its two partially ordered sets, so $\alpha(\{ \C \}) \subseteq \{ \D \}$. As $[\{ \C \}] = \{ \C \}$ and $[\{ \D \}] = \{ \D \}$, we also know that $\alpha([\{ \C \}]) \subseteq [\{ \D \}]$, or $\{ \C \} \ll \{ \D \}$. Using the first property of $\ll$ then tells us that there exists $d \in \D$ such that $\{ \tau_{1} \} \ll \{ d \}$. For the inductive step, when $| \tau | = n + 1$, we are given $\tau = \langle \dots , \tau_{n} \rangle$ and assume there exists $\upsilon = \langle \dots , \upsilon_{n} \rangle$ such that $\tau \ll \upsilon$. By the simulation property of $\ll$, we have that $M(\tau_{n}) \ll N(\upsilon_{n})$. Applying the first property of $\ll$ then states that there exists $d \in N(\upsilon_{n})$ such that $\{ \tau_{n+1} \} \ll \{ d \}$ for any $\tau_{n+1} \in M(\tau_{n})$. The concatenation of $\upsilon$ and $d$, denoted $\upsilon^{\frown}\langle d \rangle$, thus forms a valid trajectory in $\Traj(N)$ such that $| \tau | = | \upsilon^{\frown}\langle d \rangle |$ and $\tau \ll \upsilon^{\frown}\langle d \rangle$. $\qedbox$

% We claim that, for all $\tau \in \Traj(M)$, there exists a $\upsilon \in \Traj(N)$, such that $| \tau | = | \upsilon |$ and $\tau \ll \upsilon$.
