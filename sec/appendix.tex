\section{Proofs (WIP)}

\subsection{$\hat M \enc \hat N \iff \hat M \leq \hat N$}

We prove the two directions of $\hat M \enc \hat N \iff \hat M \leq \hat N$ separately.

\subsubsection{$\hat M \enc \hat N \implies \hat M \leq \hat N$:} For any $\delta \in \hat I^{+}$, let $\tau \in \hat P^{+}$ and $v \in \hat Q^{+}$ be the induced trajectories $\Driv{\hat M}{\delta}$ and $\Driv{\hat N}{\delta}$, respectively. We first prove that $\tau[k] \enc v[k]$ for each $\delta \in \hat I^{+}$ and $k \in \mathbb{N} : k < | \tau | = | v | = | \delta + 1 |$ by induction on $k$. The base case, $\tau[0] = (\top \in \hat P) \enc (\top \in \hat Q) = v[0]$, follows immediately from the definition of $\hat M \enc \hat N$. For the inductive step, assume that $\tau[k] \enc v[k]$. That also $\tau[k+1] = \hat M(\delta[k] \sqcap \tau[k]) \enc \hat N(\delta[k] \sqcap v[k]) = v[k+1]$ follows from the second property of $\tau[k] \enc v[k]$ since $\delta[k] \in \hat I$. Finally, for any $\delta \in \hat I^{+}$, that $\out{\tau[k]} \sqsubseteq \out{v[k]}$ for each $k$ follows from the first property of $\tau[k] \enc v[k]$.

\subsubsection{$\hat M \enc \hat N \Leftarrow \hat M \leq \hat N$:} Define $\enc \in \hat P \times \hat Q$ as follows:

\begin{equation*}
\bigcup_{\delta \in I^{+}} \{ (\Driv{\hat M}{\delta}[k], \Driv{\hat N}{\delta}[k]) \mid k \in \mathbb{N}, k < | \delta + 1 | \}
\end{equation*}

\noindent Here $\Driv{\hat M}{\delta}[k]$ is the k-th predicate of the trajectory induced by $\delta$. By definition of $\enc$, and that $\hat M \leq \hat N$, we thus have that any such pair of k-th predicates must have ordered outputs. That is, for any $\hat p \enc \hat q$, we have that $\out{\hat p} \sqsubseteq \out{\hat q}$. Given that $\hat p = \Driv{\hat M}{\delta}[k]$ and $\Driv{\hat N}{\delta}[k]$ for some common $k$ and $\delta \in \hat I^{+}$, we must also have that $\delta^{\frown}\hat \iota$ ($\delta$ followed by $\hat \iota$) is in $\hat I^{+}$ and thus $\hat M(\hat \iota \sqcap \hat p) = \Driv{\hat M}{\delta^{\frown}\hat \iota}[k+1] \enc \Driv{\hat N}{\delta^{\frown}\hat \iota}[k+1] = \hat N(\hat \iota \sqcap \hat q)$ for all $\hat \iota \in \hat I$. Finally, that $\hat M \enc \hat N$ follows from how $\out{\top \in \hat P} \sqsubseteq \out{\top \in \hat Q}$ is obviously true and that $\langle \hat \iota \rangle \in \hat I^{+}$ for all $\hat \iota \in \hat I$.

% for any $\hat p \enc \hat q$, we thus know that $\hat p = \tau[k]$ and $\hat q = v[k]$ for some pair of trajectories, $\tau$ and $v$, induced by some common driver $\delta$. That $\out{\hat p} \enc \out{\hat q}$ then follows from how $\hat M \leq \hat N$ implies that $\out{\tau[k]} \sqsubseteq \out{v[k]}$ for each $k$. Further, $\delta$ followed by any $\hat \iota \in \hat I$ is also a valid driver in $\hat I^{+}$. Thus, $\hat M(\hat \iota \sqcap \hat p) \enc \hat N(\hat \iota \sqcap \hat q)$ must hold for all $\hat \iota \in \hat I$ as well. That $\hat M \enc \hat N$ then follows from how $\out{\top \in \hat P} \sqsubseteq \out{\top \in \hat Q}$ is obviously true and that $\hat M(\hat \iota \sqcap \top) \enc \hat N(\hat \iota \sqcap \top)$ for every $\hat \iota \in \hat I$ since $\langle \hat \iota \rangle \in \hat I^{+}$.

% $\delta^{\frown}\hat \iota \in \hat I^{+}$
