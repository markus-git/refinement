\section{Proofs (WIP)}

We show that $\hat M \enc \hat N \iff \hat M \leq \hat N$.

\subsubsection{$\hat M \enc \hat N \implies \hat M \leq \hat N$:} For each $\delta \in \hat I^{+}$, let $\langle \hat \iota_{0}, \hat \iota_{1}, \cdots, \hat \iota_{n} \rangle \in \hat P^{+}$ be the trajectory $\Driv(\hat M)(\delta)$ and let $\langle \hat \jmath_{0}, \hat \jmath_{1}, \cdots, \hat \jmath_{n} \rangle \in \hat Q^{+}$ be $\Driv(\hat N)(\delta)$, where $n = | \delta + 1|$. We first prove that $\hat \iota_{k} \enc \hat \jmath_{k}$ for each $k \in \mathbb{N} : k < | \delta + 1 |$ and $\delta \in \hat I^{+}$ by induction on $k$. For the base case, that $\hat \iota_{0} = \top \enc \top = \hat \jmath_{0}$ follows immediately from the definition of $\hat M \enc \hat N$. For the inductive step, assume that $\hat \iota_{k} \enc \hat \jmath_{k}$. That $\hat \iota_{k+1} = \hat M(\delta[k] \sqcap \hat \iota_{k}) \enc \hat N(\delta[k] \sqcap \hat \jmath_{k}) = \hat \jmath_{k+1}$ follows from (2) of $\hat \iota_{k} \enc \hat \jmath_{k}$ since $\delta[j] \in \hat I$. Finally, given that $\hat \iota_{k} \enc \hat \jmath_{k}$ for each $k \in \mathbb{N} : k < | \delta + 1 |$, that $\out{\hat \iota_{k}} \sqsubseteq \out{\hat \jmath_{k}}$ follows naturally from (1) of $\hat \iota_{k} \enc \hat \jmath_{k}$ for each k.

% Consider an arbitrary $\delta \in \hat I^{+}$ and the two trajectories it induces, $\tau \in \Traj(\hat M)(\delta)$ and $v \in \Traj(\hat N)(\delta)$. For index $0$, prop. 2 of $\hat M \enc \hat N$ and $\delta[0] \in \hat I$ implies that $\tau[0] = \hat M(\delta[0]) = \hat M(\delta[0] \sqcap \top) \enc \hat N(\delta[0] \sqcap \top) \hat N(\delta[0]) = v[0]$, from which prop. 1 states that $\out{\tau[0]} \sqsubseteq \out{v[0]}$. For index $1$, since $\delta[1] \in \hat I$, prop. 2 of $\hat M(\tau[0]) \enc \hat N(v[0])$ implies that $\tau[1] = \hat M(\delta[1] \sqcap \tau[0]) \sqsubseteq \hat N(\delta[1] \sqcap v[0]) = v[1]$, from which prop. 1 states that $\out{\tau[1]} \sqsubseteq \out{v[1]}$. Etc. for higher indices.

\subsubsection{$\hat M \enc \hat N \Leftarrow \hat M \leq \hat N$:} 

% Prop. 1 of $\hat M \enc \hat N$, that $\top \enc \top$, is obviously true. For prop. 2, we note that every $\hat \iota_{0} \in \hat I$ also forms a valid 1-length driver $\langle \hat \iota_{0} \rangle \in \hat I^{+}$. Combined with $\hat M \leq \hat N$, we then have that $\out{\hat M(\hat \iota_{0} \sqcap \top)} = \out{\hat M(\hat \iota_{0})} \sqsubseteq \out{\hat N(\hat \iota_{0})} = \out{\hat N(\hat \iota_{0} \sqcap \top)}$ for all $\hat \iota_{0} \in \hat I$. The relation thus holds at least for the first step. But for any choice of $\hat \iota_{0}$, every following choice of $\hat \iota_{1} \in \hat I$ also forms a valid 2-length driver $\langle \hat \iota_{0}, \hat \iota_{1} \rangle \in \hat I^{+}$. We therefore have that $\out{\hat M(\hat \iota_{1} \sqcap \hat M(\hat \iota_{0}))} \sqsubseteq \out{\hat N(\hat \iota_{1} \sqcap \hat N(\hat \iota_{0}))}$ for all $\hat \iota_{0}, \hat \iota_{1} \in \hat I$ and the relation thus hold for a second step as well. Etc. for further steps. \textcolor{red}{Build this relation and show that it has the desired properties.}

% \textcolor{red}{Construct the simulation relation.}
