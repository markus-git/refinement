\section{Appendices}

\subsection{Theorem~\ref{thm:traj-refines}}

We first prove a few lemmas.

\begin{lemma} \label{lem:traj-con}
$\upsilon \in \Consequent(\rho) \wedge [\tau] \ll [\upsilon] \implies \tau \in \gamma(\Consequent(\rho))$
\end{lemma}

By property ? of $\Consequent$, $\upsilon \in \Consequent(\rho) \implies [\upsilon] \subseteq \Consequent(\rho)$. By the monotonicity of $\gamma$, $[\upsilon] \subseteq \Consequent(\rho) \implies \gamma([\upsilon]) \subseteq \gamma(\Consequent(\rho))$. By definition, $[\tau] \ll [\upsilon] \implies [\tau] \subseteq \gamma([\upsilon])$, thus $[\tau] \subseteq \gamma(\Consequent(\rho))$. $\qedbox$

\begin{lemma} \label{lem:traj-ant}
$\tau \in \gamma(\Antecedent(\rho)) \wedge [\tau] \ll [\upsilon] \implies \upsilon \in \Antecedent(\rho)$
\end{lemma}

By property ? of $\Antecedent$ and definition of $\gamma(\Antecedent)$, $\tau \in \gamma(\Antecedent(\rho)) \implies [\tau] \subseteq \gamma(\Antecedent(\rho))$. And by the monotonicity of $\alpha$, $\alpha(\tau) \subseteq \alpha(\gamma(\Antecedent(\rho)) \subseteq \Antecedent(\rho)$. By definition, $[\tau] \ll [\upsilon] \implies \alpha([\tau]) \subseteq [\upsilon]$. Because $\forall x \in \alpha([\tau]) : [x] = [\upsilon]$, property ? of $\Antecedent$ and implies that $[\upsilon] \subseteq \Antecedent(\rho) \implies \upsilon \in \Antecedent(\rho)$. $\qedbox$

Given a $\tau \in \Traj(M)$ and $\rho \in \Runs(\gamma(A))$, such that $| \tau | = | \rho |$ and $\tau_{n} \in \gamma(\Antecedent(\rho_{n}))$ for all $n \in \mathbb{N} : n < | \tau |$, we must show that $\tau_{n} \in \gamma(\Consequent(\rho_{n}))$. By the refinement assumption, there must exist a $\upsilon \in \Traj(N)$ such that $| \tau | = | \upsilon | = | \rho |$ and $[\tau_{n}] \ll [\upsilon_{n}]$. By lemma~\ref{lem:traj-ant}, $\upsilon_{n} \in \Antecedent(\rho_{n})$, and by the assumption, $\upsilon_{n} \in \Consequent(\rho_{n})$. Finally, lemma~\ref{lem:traj-con} then states that $\tau_{n} \in \gamma(\Consequent(\rho_{n}))$. $\qedbox$

\subsection{Theorem~\ref{thm:traj-equal-set}}

We prove each direction separately.

$(\Rightarrow)$ : Text.

$(\Leftarrow)$ : Text.

%% We first prove a few lemmas.

%% \begin{lemma} \label{lem:traj-part}
%% $\tau \in \Traj(M) \iff \tau \in \Traj(M)(\In(\tau))$
%% \end{lemma}

%% \begin{equation*}
%% \Traj(M)(\In(\tau)) = \{ \tau \in C^{+} \mid \tau \in \Traj(M) : \In(\tau) = \In(\tau) \} = \Traj(M)
%% \end{equation*}

%% \begin{lemma} \label{lem:traj-all-in}
%% $\bigcup_{\delta \in \In[C]^{+}} \{ \tau_{0} \mid \tau \in \Traj(M)(\delta) \} = C$
%% \end{lemma}

%% \todo{Input equality forms an equivalence class for trajectories, and the union of each class cover $\Traj(M)$. Further, as the initial element of each trajectory is unconstrained, their union must contain every possible configuration.}

%% %% \begin{align*}
%% %% \bigcup_{i \in \In[C]} \Traj(M)(\langle i \rangle) & = \bigcup_{i \in \In[C]} \{ \langle c \rangle \in C^{1} \mid \forall \langle c \rangle \in \Traj(M) : \In(c) = i \} \\ & = \bigcup_{i \in \In[C]} \{ \langle c \rangle \in C^{1} \mid \forall c \in C : \In(c) = i \} = \langle C \rangle
%% %% \end{align*}

%% \begin{lemma} \label{lem:traj-all-out}
%% $\bigcup_{\delta \in \In[C]^{+}} \{ \bigcup_{\varepsilon \in \Out[C]^{+}} \{ \tau_{0} \mid \tau \in \Traj(M)(\delta) : \Out(\tau) = \varepsilon \} \} = C$
%% \end{lemma}

%% \todo{Similar reasoning as above, but partitioned twice.}

%% \begin{lemma} \label{lem:cl-union}
%% $c_{1} \ll d_{1} \wedge c_{2} \ll d_{2} \implies (c_{1} \cup c_{2}) \ll (d_{1} \cup d_{2})$
%% \end{lemma}

%% \todo{Text.}

%% \begin{lemma} \label{lem:in-iff-sim}
%% $M \Refines_{in} N \iff M \SetRefines N$
%% \end{lemma}

%% We prove each direction of the theorem separately:

%% $(\Rightarrow)$ : We define the relation $\ll \, \subseteq \pow(C) \times \pow(D)$ as follows:

%% \begin{equation*}
%% \close{\cup} \{ ( \{ \tau_{| \delta | - 1} \} , \{ \upsilon_{| \delta | - 1} \} ) \mid \forall \delta \in \In[C]^{+} : \forall \tau \in \Traj(M)(\delta) : \forall \upsilon \in \Traj(N)(\delta) : \Out(\tau) = \Out(\upsilon) \}
%% \end{equation*}

%% \noindent where $\tau_{| \delta | - 1}$ and $\upsilon_{| \delta | - 1}$ denote the last element of both sequences.

%% We show that $\ll$ is a simulation relation on the visible elements in $\pow(C)$ and $\pow(D)$. Firstly, for any $c \ll d$, both $c$ and $d$ are unions of a finite number of $\tau \in \Traj(M)(\delta)_{| \delta | - 1}$ and $\upsilon \in \Traj(N)(\delta)_{| \delta | - 1}$ pairs with some common $\delta$. By definition of $\ll$, we know that $[\tau]_{\Out} = [\upsilon]_{\Out}$ for each such $\tau$ and $\upsilon$, and thus $\Out[c] \subseteq \Out[d]$. Secondly, the concatenation of any such $\delta$ and each $i \in \In[C]$ forms another driver $\delta^{\frown} \langle i \rangle \in \In[C]^{+}$. \todo{...}

%% % exist a $\upsilon' \in \Traj(N)(\delta^{\frown} \langle i \rangle)_{| \delta |}$ for every $\tau' \in \Traj(M)(\delta^{\frown} \langle i \rangle)_{| \delta |}$ such that $[\tau']_{\Out} = [\upsilon']_{\Out}$.

%% Finally, to show that $M \SetRefines N$ we must show that $C \ll D$. \todo{By lemma~\ref{lem:traj-all-in} and the forall-quantification over drivers, we know that every member of $C$ is considered. Further, as $\Out[C] = \Out[D]$ and every output of $C$ is considered, every member of $D$ must also be considered.}

%% % last elements in some pair of trajectories with equal outputs. Thus any union of such elements must have equal outputs as well, which implies that $\Out[c] \subseteq \Out[d]$ holds. Lastly, for every driver used to generate the elements that make out $c$ and $d$, extending them with any input produces another, valid driver. Therefore any 

%% $(\Leftarrow)$ : Text.

%% \begin{corollary}
%% $M \Refines N \iff M \SetRefines N$
%% \end{corollary}

%% Follows immediately from lemma~\ref{lem:ref-iff-in} and~\ref{lem:in-iff-sim}.

