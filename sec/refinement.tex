\section{System refinement}

% we then we say that $\hat M$ \textit{refines} $\hat N$ if every \textit{externally visible behaviour} allowed by $\hat M$ is also allowed by $\hat N$, regardless of any initial configurations.

Consider another fixed, but arbitrary, circuit model $N \in \mathcal{P}(C') \rightarrow \mathcal{P}(C')$ such that $M$ and $N$ have the same number of inputs and outputs but can differ internally. Let the abstract predicate transformer $\hat N \in \hat Q \rightarrow \hat Q$ be an abstract interpretation of $N$, where $\hat Q$ is an abstact predicate for $\mathcal{P}(C)$. In the previous sections, exactly what abstract predicates like $\hat Q$ are, were not important. In order to reason about refinement, however, we  make a distinction between their \textit{external} and \textit{internal} elements.

Specifically, let the \textit{externally visible} parts of an abstract predicate $\hat Q$ be the subsets given by two projections, $\InProj{\cdot}$ and $\OutProj{\cdot}$, identifying the ``inputs'' and ``outputs'' of $\hat Q$, respectively. Further, let $\Out \in \hat Q \rightarrow \OutProj{\hat Q}$ be a mapping that takes each $\hat q \in \hat Q$ to its visible outputs, and let $\In \in \InProj{\hat Q} \rightarrow \hat Q$ be another mapping that lifts each $\hat \iota \in \InProj{\hat Q}$ into $\hat Q$ such that $\Out(\hat q) \sqsubseteq \hat \iota \iff \hat q \sqsubseteq \In(\hat \iota)$ for all $\hat q \in \hat Q$. With a slight abuse of notation, overload both projections and mappings to accept predicates from $\hat P$ and note that $| \OutProj{\hat P} | = | \OutProj{\hat Q} |$ and $| \InProj{\hat P} | = | \InProj{\hat Q} |$.

\todo{This $\In$ and $\Out$ business reminds me of Galois connection.}

Let $\cc \, \subseteq \hat P \times \hat Q$ be a Galois connection such that for all $\hat Q' \subseteq \hat Q$ and $\hat P' \subseteq \hat P$:

\begin{equation*}
\forall \hat q \in \hat Q' : \forall \hat p \in \hat P' : \hat q \cc \hat p \iff \sqcup \hat Q' \cc \sqcap \hat P'
\end{equation*}

\noindent Like the earlier Galois connections, the relation $\cc$ can intuitively be thought of as an extension of the internal orderings of $\hat P$ and $\hat Q$ to an ordering between the two abstract predicates. \todo{Text.}

A \textit{driver} $\langle \hat \iota_{0}, \ldots \rangle \in \InProj{\hat Q}^{+}$ for $\hat N$ is a nonempty sequence of inputs which induces a \textit{trajectory} $\langle \tau_{0}, \ldots, \tau_{| \delta | + 1} \rangle \in \hat Q^{+}$, where $\tau_{0} = \top$ and $\tau_{i+1} = \hat N(\In(\hat \iota_{i}) \sqcap \tau_{i})$ for all $i \in \mathbb{N} : i < | \delta |$; the trajectory induced in $\hat N$ by a driver $\delta$ is denoted by $\Driv{\hat N}{\delta}$.

We view circuits as transducers and say that $\hat M$ refines $\hat N$ if:

\begin{equation*}
\forall \delta \in ?^{+} : \out{\Driv{\hat M}{\delta}} \cc \out{\Driv{\hat N}{?(\delta)}}
\end{equation*}

% ... to its visible outputs $\Out(\hat Q) \in \OutProj{\hat Q}$.
% We extend $\out{\cdot}$ to sequences component-wise and, ...

\subsection{Old stuff, trying to make it more formal}

A \textit{driver} of $\hat M$ and $\hat N$ is a nonempty sequence of inputs, $\delta \in \hat I^{+}$, and induces a trajectory $\tau$ in $\hat M$ (resp. $\hat N$) where $\tau[0] = \top$ and $\forall i \in \mathbb{N} : i < | \delta | \implies \tau[i+1] = \hat M(\delta[i] \sqcap \tau[i])$; the trajectory induced by a driver $\delta$ in $\hat M$ is denoted by $\Driv{\hat M}{\delta}$.

Intuitively, if $\hat M$ produces the same, or at least more specified, outputs than $\hat N$ for all possible drivers, then every visible behaviour of $\hat M$ is covered by $\hat N$. We thus say that $\hat M$ refines $\hat N$, denoted by $\hat M \leq \hat N$, iff:

% \todo{Footnote that ``Traj'' comes from trajectories in prev. papers?}

% \todo{We have to use $=$ for our trajectories since $\sqsubseteq$ allows one to pick bad states for unconstrained wires.}

\begin{equation*}
\forall \delta \in \hat I^{+} : \out{\Driv{\hat M}{\delta}} \sqsubseteq \out{\Driv{\hat N}{\delta}}
\end{equation*}

\todo{Example!}

\subsubsection{Simulation} Let $\enc \in \hat P \times \hat Q$ be a simulation relation such that $\hat p \enc \hat q$ implies (1) $\out{\hat p} \sqsubseteq \out{\hat q}$ and (2) $\hat M(\hat \iota \sqcap \hat p) \enc \hat N(\hat \iota \sqcap \hat q)$ for all inputs $\hat \iota \in \hat I$. We extend this relation to $\hat M$ and $\hat N$ such that $\hat M \enc \hat N$ iff their top elements are related, $(\top \in \hat P) \enc \, (\top \in \hat Q)$.

\todo{This relation is equivalent to the earlier notion of refinement: $\hat M \leq \hat N \iff \hat M \enc \hat N$.}

\todo{We can have trajectories defined by the simulation relation as well.}

% (read as $\hat p$ ``encodes'' $\hat q$)

\todo{Reference proof in appendix.}

\todo{Explain difference with Bryant's version?}

Let $G \in S \rightarrow (\hat Q \rightarrow \hat Q)$ and $\mathcal{G} \in (S \rightarrow \hat Q) \rightarrow (S \rightarrow \hat Q)$ be the duals of $F$ and $\mathcal{F}$ in $\hat N$, respectively. Further, let $\Psi_{*}$ be the least fix point of $\Psi = \mathcal{G}(\Psi)$ and the dual of $\Phi_{*}$ in $\hat N$.

\begin{lemma}
$\forall s \in S : \Psi_{*}(s) \enc \Phi_{*}(s)$
\end{lemma}

\subsubsection{Trajectory} A trajectory assertion $\hat A = (S,s_{0},R,\pi_{a},\pi_{c})$ for $\hat N$ where antecedents only mention inputs, $\pi_{a} \in S \rightarrow \hat I$, and consequents only mention outputs, $\pi_{c} \in S \rightarrow \hat O$, is referred to as an \textit{external trajectory assertion}. Intuitively, a satisfied external trajectory assertion is property of $N$ that must hold regardless of its internal state.

Given that $\Phi_{*}(s) \enc \Psi_{*}(s)$, and thus $\out{\Phi_{*}(s)} \sqsubseteq \out{\Psi_{*}(s)}$, for all $s \in S$, it follows that $\Phi_{*}(s) \sqcap \pi_{a}(s) \sqsubseteq \pi_{c}(s) \implies \Psi_{*}(s) \sqcap \pi_{a}(s) \sqsubseteq \pi_{c}(s)$.

\begin{theorem}
Given two abstract predicate transformers, $\hat M$ and $\hat N$, and an external trajectory assertion $\hat A$ for $\hat N$. If $\hat M \preceq \hat N$, then $\hat N \models \hat A$ implies that $\hat M \models \hat A$.
\end{theorem}

\begin{corollary}
Given two abstract predicate transformer, $\hat M$ and $\hat N = \prod \hat n_{i}$, and a trajectory assertion $\hat A$ for $\hat N$. Let $\hat N[\hat n_{i} = \hat M]$ be $\hat N$ where $\hat n_{i}$ is replaced with $\hat M$. If $\hat M \enc \hat n_{i}$ for some index $i$, then $\hat N \models \hat A$ implies that $\hat N[\hat n_{i} = \hat M] \models \hat A$.
\end{corollary}

% LocalWords:  consequents
