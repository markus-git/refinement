\section{Refinement}

\subsection{Set-Theoretic refinement}

Consider another fixed, but arbitrary, circuit model $N \in \pow(D) \rightarrow \pow(D)$, where $D$ is a non-empty and finite set of configurations. Further, let there be a Galois connection between predicates $(C, \subseteq)$ and $(D, \subseteq)$ given by means of a binary relation $\ll \, \subseteq \pow(C) \times \pow(D)$, where $c \ll d$ is read as ``c can be approximated by d''. We derive $\ll$ from either of the usual functions for abstraction $\alpha \in \pow(C) \rightarrow \pow(D)$ and concretisation $\gamma \in \pow(D) \rightarrow \pow(C)$ as follows:

\begin{equation*}
c \ll d \iff \alpha(c) \subseteq d \vsep c \ll d \iff c \subseteq \gamma(d)
\end{equation*}

\noindent Here $\subseteq$ on the $\alpha$-derivation side is the inclusion order of $\pow(D)$, and on the $\gamma$-derivation side $\subseteq$ is the inclusion order of $\pow(C)$. Intuitively, the relation $\ll$ acts as an extension of the orderings inside $\pow(C)$ and $\pow(D)$ to one between them.

Exactly what configurations such as $C$ and $D$ are, were not important previously. To reason about refinement, which relates the visible behaviour of circuits, we make a distinction between their external and internal elements. Let the visible elements of a configuration in $C$ be identified by a function $\Vis \in C \rightarrow \pow(C)$, where $\Vis(c)$ denotes the configurations which are \textit{visually} equal to a $c \in C$. Further, let $\Vis(\cdot)$ induce an equivalence relation $\sim$ on configurations in $C$, such that $c \sim c' \iff \Vis(c) = \Vis(c')$; the equivalence class of $c$ under $\sim$, denoted by $[c]_{\sim}$, is defined as $\{ c' \in C \mid c' \sim c \}$. With a slight abuse of notation, we extend $\Vis(\cdot)$ to sequences component-wise and overload it to accept configurations in $D$.

We now formalize an intuition of whether $M$ \textit{refines} $N$, denoted by $M \Refines N$:

\begin{equation*}
\forall \tau \in \Traj(M) : \exists \upsilon \in \Traj(N) : | \tau | = | \upsilon | \wedge \Vis(\tau) \ll \Vis(\upsilon)
\end{equation*}

\noindent where two sequences $\langle \Vis(\tau_{0}), \ldots \rangle \ll \langle \Vis(\upsilon_{0}), \ldots \rangle$ iff $\Vis(\tau_{n}) \ll \Vis(\upsilon_{n})$ for all $n \in \mathbb{N} : n < | \tau | = | \upsilon |$. In other words, $M \Refines N$ states that, for every sequence of configurations $\tau$ permitted by $M$, there must exist a sequence $\upsilon$ for $N$ which ``covers'' the visual behaviour of $\tau$.

Recall that a trajectory assertion for $N$ is a quintuple $A = (S, s_{0}, R, \Antecedent, \Consequent)$, where $\Antecedent \in S \rightarrow \pow(D)$ and $\Consequent \in S \rightarrow \pow(D)$ label each $s \in S$ with its antecedents and consequents, respectively. If these antecedents and consequents are limited to partitions of visible elements in $D$, i.e. $d \in \Antecedent(s) \implies [d]_{\sim} \subseteq \Antecedent(s)$ for all $s \in S$, then we refer to $A$ as a \textit{visible trajectory assertion} and denote it by $A_{\textrm{vis}}$. Define $\gamma(A) = (S, s_{0}, R, \gamma(\Antecedent), \gamma(\Consequent))$, where $\gamma(\Antecedent) = \lambda s \in S : \gamma(\Antecedent(s))$ and $\gamma(\Consequent) = \lambda s \in S : \gamma(\Consequent(s))$. We can now state that refinement between $M$ and $N$ entails that satisfaction of any $A_{\textrm{vis}}$ by $N$ implies that $M$ satisfies $\gamma(A_{\textrm{vis}})$ as well:

\begin{theorem} \label{thm:traj-refines}
$M \Refines N \implies (N \models A_{\textrm{vis}} \implies M \models \gamma(A_{\textrm{vis}}))$
\end{theorem}

The above definition of refinement can be equivalently formulated as a simulation relation. We say that $M$ refines $N$ by \textit{set-theoretic refinement}, denoted by $M \SetRefines N$, iff $C$ can be approximated by $D$, i.e. $C \ll D$; $\ll$ is a \textit{visual approximation}, i.e. $c \ll d \implies \alpha(\Vis[c]) \subseteq \Vis[d]$; and $\ll$ is a \textit{simulation relation} from $\pow(C)$ to $\pow(D)$, i.e. $c \ll d \implies M(c) \ll N(d)$. That $\ll$ is a simulation relation can also be stated in terms of the usual functions for abstraction $\alpha$ and concretisation $\gamma$, as we did in the previous definition of abstract predicate transformers.

\begin{theorem} \label{thm:traj-equal-set}
$M \Refines N \iff M \SetRefines N$
\end{theorem}
