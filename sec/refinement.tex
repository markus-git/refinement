\section{Refinement}

\subsection{Set-Theoretic refinement}

Consider another fixed, but arbitrary, circuit model $N \in \Pow(D) \rightarrow \Pow(D)$, where $D$ is a non-empty and finite set of configurations which intersects the earlier set $C$. Exactly what configurations such as $C$ and $D$ are, were not important previously. To reason about refinement, which relates the visible behaviour of circuits, we make a distinction between their external and internal elements.

% The rationale is that refinement relates the visible behaviour of circuits.

Let the visible elements of a configuration in $C$ be identified by two projections, $\In \in C \rightarrow C$ and $\Out \in C \rightarrow C$, where $[c]_{\In}$ denotes the \textit{inputs} and $[c]_{\Out}$ the \textit{outputs} of any $c \in C$. With a slight abuse of notation, we overload both $\In$ and $\Out$ to accept configurations in $D$ and extend them to sequences component-wise. \todo{Doing so allows the following, classical intuition of refinement in terms of trajectories for $M$ and $N$:}

% the sets of all possible inputs and outputs in $M$ are given by the images $\In[\Pow(C)] = \{ [c]_{\In} \mid c \in \Pow(C) \}$ and $\Out[\Pow(C)]$.

\begin{equation*}
\forall \tau \in \Traj(M) : \exists v \in \Traj(N) : | \tau | = | v | \wedge [\tau]_{\In} = [v]_{\In} \wedge [\tau]_{\Out} = [v]_{\Out}
\end{equation*}

% The above definition of refinement is equivalent to one expressed as a simulation relation. $\SetRefine \in \pow(C) \times \pow(D)$ iff $p \SetRefine q$ implies that $\Out(p) \subseteq \Out(q)$ and $M(p \cap i) \SetRefine N(q \cap i)$ for all $i \in \In[\pow(C)]$. $M \SetRefine N$ iff $C \SetRefine D$.
