\section{System refinement}

Consider another fixed, but arbitrary, circuit model $N \in \mathcal{P}(D) \rightarrow \mathcal{P}(D)$ where its set configurations $D$ is non-empty and finite. Exactly what configurations such as $D$ are, were not important previously. To reason about refinement of models, however, we make a distinction between their external and internal elements. The rationale is that refinement relate visible behaviours of circuit models, it should not matter whether their internal elements can be aligned or not.

Let the visible components visible elements of a configurations $C$ be identified by two projection mappings, $\Out \in \mathcal{P}(C) \rightarrow \mathcal{P}(C)$ and $\In \in \mathcal{P}(C) \rightarrow \mathcal{P}(C)$, identifying its ``inputs'' and ``outputs'', respectively. The set of all possible inputs is the image $\In[\mathcal{P}(C)] = \{ \In(p) \mid p \in \mathcal{P}(C) \}$, which we denote by $I$. Note that, since $M$ cannot control its input signals, the transitions of $M$ in response to some input $i \in I$ are given by $M(p \cap i)$, for any initial state $p \in \mathcal{P}(C)$.

A \textit{trajectory} of $M$ is a non-empty sequence of configurations, $\tau \in C^{+}$, induced by its response to a \textit{driver} $\delta \in I^{+}$. Specifically, $\tau_{0} = C$ and $n \in \mathbb{N} : n < | \delta | \implies \tau_{n+1} = M(\tau_{n} \cap \delta_{n})$; the trajectory of $M$ induced by a $\delta$ is denoted by $M_{\delta}$ and is prefix-closed. As $C$ is finite, the set of all possible trajectories of $M$ fully capture its behaviour for starting in the arbitrary initial state.

