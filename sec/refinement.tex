\section{System refinement (WIP)}

Consider another fixed, but arbitrary, circuit $N$ such that configurations of $M$ and $N$ have the same externally visible elements but can differ internally. Let $\hat N \in \hat Q \rightarrow \hat Q$ be an abstract predicate transformer of $N$, we then we say that $\hat M$ \textit{refines} $\hat N$ if every \textit{externally visible behaviour} allowed by $\hat M$ is also allowed by $\hat N$, regardless of any initial configurations.

\textcolor{red}{$M$ always sets it $\hat I$ to $\top$ and doesn't really work for zero-delay circuits.}

\textcolor{red}{The value domain for inputs and outputs must be the same so they are comparable. Not just that they have the ``same'' inputs and output elements.}

\textcolor{red}{Example?}

\subsubsection{Drivers} Let the \textit{externally visible} parts of an abstract predicate $\hat P$ be the subsets given by two projections, $\InProj$ and $\OutProj$, identifying the ``inputs'' and ``outputs'' of $\hat P$, respectively. Further, let $\out{\cdot} \in \hat P \rightarrow \hat O$ be a mapping that takes each $\hat p \in \hat P$ to its visible outputs $\out{\hat p} \in \hat O$; $\out{\cdot}$ is extended to sequences component-wise. With a slight abuse of notation, we overload both projections and the mapping to also accept predicates from $\hat Q$ and note that $\InProj(\hat P) = \InProj(\hat Q) = \hat I$ and $\OutProj(\hat P) = \OutProj(\hat Q) = \hat O$ since $M$ and $N$ share inputs and outputs.

A \textit{driver} of $\hat M$ and $\hat N$ is a nonempty sequence of inputs, $\delta \in \hat I^{+}$, and induces a trajectory $\tau$ in $\hat M$ (resp. $\hat N$) where $\tau[0] = \top$ and $\forall i \in \mathbb{N} : 0 < i < | \delta + 1 | \implies \tau[j] = \hat M(\delta[j-1] \sqcap \tau[j-1])$; the trajectory induced by a driver $\delta$ in $\hat M$ is denoted by $\Driv(\hat M)(\delta)$. Intuitively, if $\hat M$ produces the same, or at least more specified, outputs than $\hat N$ for all possible drivers, then every visible behaviour of $\hat M$ is covered by $\hat N$. We thus say that $\hat M$ refines $\hat N$, denoted by $\hat M \leq \hat N$, iff:

\begin{equation*}
\forall \delta \in \hat I^{+} : \out{\Driv(\hat M)(\delta)} \sqsubseteq \out{\Driv(\hat N)(\delta)}
\end{equation*}

\textcolor{red}{Example!}

\subsubsection{Simulation} Let $\enc \in \hat P \times \hat Q$ be a simulation relation such that $\hat p \enc \hat q$ implies (1) $\out{\hat p} \sqsubseteq \out{\hat q}$ and (2) $\hat M(\hat \iota \sqcap \hat p) \enc \hat N(\hat \iota \sqcap \hat q)$ for all inputs $\hat \iota \in \hat I$. We extend this relation to $\hat M$ and $\hat N$ such that $\hat M \enc \hat N$ iff their top elements are related, $\top \enc \top$. We then simplify refinement thus: $\hat M \leq \hat N \iff \hat M \enc \hat N$.

\textcolor{red}{Example!}

% (read as $\hat p$ ``encodes'' $\hat q$)

\subsubsection{Trajectory} A trajectory assertion $\hat A = (S,s_{0},R,\pi_{a},\pi_{c})$ for $\hat N$ where antecedents only mention inputs, $\pi_{a} \in S \rightarrow \hat I$, and consequents only mention outputs, $\pi_{c} \in S \rightarrow \hat O$, is referred to as an \textit{external trajectory assertion}. Intuitively, a satisfied external trajectory assertion is property of $N$ that must hold regardless of its internal state.

Let $G \in S \rightarrow (\hat Q \rightarrow \hat Q)$ and $\mathcal{G} \in (S \rightarrow \hat Q) \rightarrow (S \rightarrow \hat Q)$ be the duals of $F$ and $\mathcal{F}$ in $\hat N$, respectively. Further, let $\Psi_{*}$ be the least fix point of $\Psi = \mathcal{G}(\Psi)$ and the dual of $\Phi_{*}$ in $\hat N$. \textcolor{red}{Fix for red below.}

For all below, assume $\hat M \enc \hat N$.

(1): $\hat p \enc \hat q \implies \forall s \in S : F(s)(\hat p) \enc G(s)(\hat q)$. $F(s)(\hat p) = \hat M(\pi_{a}(s) \sqcap \hat p)$ and $G(s)(\hat q) = \hat N(\pi_{a}(s) \sqcap \hat q)$ where $\pi_{a}(s) \in \hat I$. By prop. 2 of $\hat p \enc \hat q$, we thus know that $F(s)(\hat p) \enc G(s)(\hat q)$.

(2): $\Phi_{*}(s) \enc \Psi_{*}(s)$ for all $s \in S$. By induction on $n$, like below. Base case and case when $s = s_{0}$ are both trivial. For the inductive step, assume that $\Phi_{n}(s) \enc \Psi_{n}(s)$ for all $s \in S$. For any $s \neq s_{0}$, we have that $\Phi_{n+1} = \sqcup \{ F(s')(\Phi_{n}(s')) \mid (s',s) \in R \}$. By (1) we know that ...

If $\hat M \enc \hat N$, we claim that $\out{\Phi_{*}(s)} \sqsubseteq \out{\Psi_{*}(s)}$ for all $s \in S$. Since $\Phi_{*}(s) = \lim \, \Phi_{n}(s)$ and $\Psi_{*}(s) = \lim \, \Psi_{n}(s)$, it suffices to prove that $\out{\Phi_{n}(s)} \sqsubseteq \out{\Psi_{n}(s)}$ for all $s \in S$ and $n \in \mathbb{N}$. We do so by induction on $n$. The base case is trivially true ($\out{\bot} \sqsubseteq \out{\bot}$). For the inductive step, assume $\out{\Phi_{n}(s)} \sqsubseteq \out{\Psi_{n}(s)}$ for all $s \in S$. That $\out{\Phi_{n+1}(s)} \sqsubseteq \out{\Psi_{n+1}(s)}$ for $s_{0}$ is also trivially correct ($\out{\top} \sqsubseteq \out{\top}$). For any $s \neq s_{0} \in S$, we have $\out{\Phi_{n+1}(s)} = \out{\sqcup \{ F(s')(\Phi_{n}(s')) \mid (s',s) \in R \}} \sqsubseteq \out{\sqcup \{ F(s')(\Psi_{n}(s')) \mid (s',s) \in R \}}$. \textcolor{red}{I would like to claim that this is $\sqsubseteq \out{\sqcup \{ G(s')(\Psi_{n}(s')) \mid (s',s) \in R \}} = \out{\Psi_{n+1}}$. But I'm missing a relation between $F$ and $G$, which should come from $\enc$.}

\begin{theorem}
Given two abstract predicate transformers $\hat M$ and $\hat N$ such that $\hat M \preceq \hat N$, and a external trajectory assertion $\hat A$ for $\hat N$, then $\hat N \models \hat A \implies \hat M \models \hat A$.
\end{theorem}

% LocalWords:  consequents
