\section{System refinement}

Consider another fixed, but arbitrary, circuit model $N \in \mathcal{P}(C') \rightarrow \mathcal{P}(C')$ and let $\hat N \in \hat Q \rightarrow \hat Q$ be an abstract interpretation of it; $\hat Q$ is an abstract predicate such that there is a Galois connection between $\mathcal{P}(C')$ and itself. In the previous sections, exactly what abstract predicates like $\hat Q$ are, were not important. In order to reason about refinement, however, we need to make a distinction between their \textit{external} elements. Specifically, we say that $\hat M$ refines $\hat N$ if every externally visible behaviour of $\hat M$ is allowed by $\hat N$ while assuming nothing about their initial configurations.

Let the \textit{externally visible} parts of an abstract predicate $\hat P$ be those given by two idempotent mappings, $\In \in \hat P \rightarrow \hat P$ and $\Out \in \hat P \rightarrow \hat P$, which identify the ``inputs'' and ``outputs'' of $\hat P$, respectively. An \textit{input} of $\hat P$ (resp. \textit{output}) is thus a predicate $\hat \iota \in \hat P$ such that $\hat \iota = \In(\hat \iota)$ (resp. $\hat \omicron = \Out(\hat \omicron) \in \hat P$), and the response of $\hat M$ in state $\hat p$ to an input $\hat \iota$ is given by $\hat M(\hat \iota \sqcap \hat p)$. With a slight abuse of notation, we overload both $\In(\cdot)$ and $\Out(\cdot)$ to also accept predicates from $\hat Q$.

% \textit{outputs} of $\hat P$ are defined similarly with $\Out(\cdot)$.

% Together, $\In(\cdot)$ and $\Out(\cdot)$ identify the external part of $\hat P$: $\Ext(\hat p) = \In(\hat p) \sqcap \Out(\hat p)$ for all $\hat p \in \hat P$.

% Specifically, let the \textit{externally visible} parts of an abstract predicate $\hat P$ be the subsets given by two projections, $\In(\cdot)$ and $\Out(\cdot)$, identifying the subsets of $\hat P$ that contain its ``inputs'' and ``outputs'', respectively. Further, let $\Out \in \hat P \rightarrow \OutProj{\hat P}$ and $\In \in \hat P \rightarrow \InProj{\hat P}$ be the projections viewed as idempotent mappings, and 

% With a slight abuse of notation, we overload both projections to accept predicates from $\hat Q$ and note that $| \OutProj{\hat P} | = | \OutProj{\hat Q} |$ and $| \InProj{\hat P} | = | \InProj{\hat Q} |$.

% Further, let $\Out \in \hat P \rightarrow \OutProj{\hat P}$ and $\In \in \hat P \rightarrow \InProj{\hat P}$ be invertible mappings that takes each $\hat p \in \hat P$ to its visible outputs and inputs, respectively, such that $\In^{-1}(\hat p) \sqsubseteq \hat \iota \iff \hat p \sqsubseteq \In(\hat \iota)$ for all $\hat p\in \hat P$. With a slight abuse of notation, we overload these projections and mappings to accept predicates from $\hat Q$

% ... note that $| \OutProj{\hat P} | = | \OutProj{\hat Q} |$ and $| \InProj{\hat P} | = | \InProj{\hat Q} |$.

% ... to its visible outputs $\Out(\hat Q) \in \OutProj{\hat Q}$.

% We extend $\out{\cdot}$ to sequences component-wise and, ...

Let $\cc \, \subseteq \hat P \times \hat Q$ be a Galois connection such that for all $\hat P' \subseteq \hat P$ and $\hat Q' \subseteq \hat Q$:

\begin{equation*}
\forall \hat p \in \hat P' : \forall \hat q \in \hat Q' : \hat p \cc \hat q \iff \sqcup \hat P' \cc \sqcap \hat Q'
\end{equation*}

\noindent Like the earlier Galois connection, $\cc$ can intuitively be thought of as an extension of the orderings inside $\hat P$ and $\hat Q$ to an ordering between them. \todo{The abstraction and concretisation functions, $\alpha \in \hat P \rightarrow \hat Q$ and $\gamma \in \hat Q \rightarrow \hat P$, are derived from $\cc$ as before. We note that $\gamma$ is monotone, preserves top and distributes over arbitrary meet, i.e. $\gamma(\sqcap \hat Q) = \sqcap \{ \gamma(\hat q) \in \hat P \mid \hat q \in \hat Q\}$. Similarly, $\alpha$ is monotone, preserves bottom and distributes over arbitrary join}.

The binary relation $\cc$ is a \textit{visible simulation} between $\hat P$ and $\hat Q$ if $\hat p \cc \hat q$ implies (1) $\Out(\hat p) \sqsubseteq \gamma(\Out(\hat q))$ and (2) $\hat M(\hat \iota \sqcap \hat p) \cc \hat N(\alpha(\hat \iota) \sqcap \hat q)$ for all inputs $\hat \iota \in \hat P$. The abstract model $\hat M$ then refines $\hat N$ if the top element of $\hat P$ visibly simulates the top of $\hat Q$, i.e. $(\top \in \hat P) \cc \, (\top \in \hat Q)$.

% The abstraction and concretisation functions, $\PQ \in \hat P \rightarrow \hat Q$ and $\QP \in \hat Q \rightarrow \hat P$, are derived from $\cc$. We note that $\QP$ is montone, preserves top and distributes over arbitrary meet, i.e. $\QP(\sqcap \hat Q) = \sqcap \{ \QP(\hat q) \in \hat P \mid \hat q \in \hat Q\}$. Similarly, $\PQ$ is also montone, preserves bottom and distributes over arbitrary join.

% A \textit{driver} $\delta \in \hat P^{+}$ for $\hat M$ is a nonempty sequence of abstract predicates $\langle \hat \iota_{0}, \ldots \rangle$ such that $\hat \iota_{i} = \In(\hat \iota_{i})$ for all $i \in \mathcal{N} : i < | \delta |$. Every driver $\delta$ induces a \textit{trajectory} $\langle \tau_{0}, \ldots, \tau_{| \delta | + 1} \rangle \in \hat P^{+}$ in $\hat M$, where $\tau_{0} = \top$ and $\tau_{i+1} = \hat N(\hat \iota_{i} \sqcap \tau_{i})$ for all $i \in \mathbb{N} : i < | \delta |$; the trajectory induced in $\hat M$ by a driver $\delta$ is denoted by $\Driv{\hat N}{\delta}$.

% Intuitively, $\hat M$ refines $\hat N$ if the outputs of their induced trajectories are ordered for all possible input sequences:

% We view circuits as transducers and say that $\hat M$ refines $\hat N$ if:

%% \begin{equation*}
%% \forall \delta \in \hat P^{+} : \Out(\Driv{\hat M}{\delta}) \sqsubseteq \gamma(\Out(\Driv{\hat N}{\alpha(\delta)}))
%% \end{equation*}

%% Let $\enc \in \hat P \times \hat Q$ be a simulation relation such that $\hat p \enc \hat q$ implies:

%% \begin{enumerate}
%% \item $\Out(\hat p) \sqsubseteq \gamma(\Out(\hat q))$ and
%% \item $\hat M(\hat \iota \sqcap \hat p) \enc \hat N(\alpha(\hat \iota) \sqcap \hat q)$ for all $\hat \iota \in \hat P$ such that $\hat \iota = \In(\hat \iota)$.
%% \end{enumerate}

%% \noindent We extend this relation to $\hat M$ and $\hat N$ and say that $\hat M \enc \hat N \iff (\top \in \hat P) \enc \, (\top \in \hat Q)$.

\subsection{Examples}

\begin{align*}
\In(\langle \hat p_{0}, \hat p_{1}, \hat p_{2} \rangle) = \langle \hat p_{0}, \hat p_{1}, \X \rangle && \Out(\langle \hat p_{0}, \hat p_{1}, \hat p_{2} \rangle) = \langle \X, \X, p_{2} \rangle
\end{align*}

% LocalWords:  consequents
