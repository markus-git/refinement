\section{Refinement}

\subsection{Set-Theoretic refinement}

Consider another fixed, but arbitrary, circuit model $N \in \pow(D) \rightarrow \pow(D)$, where $D$ is a non-empty and finite set of configurations which intersects the earlier set $C$. Exactly what configurations such as $C$ and $D$ are, were not important previously. To reason about refinement, however, we need to make a distinction between their external and internal elements. The rationale is that refinement relates the visible behaviour of circuits. \todo{Unnecessary to assume that internal states can be aligned.}

% internal states is of no real significance if their external behaviours match.

% , i.e. we assume $C \cap D \neq \emptyset$

% Restricting refinement to models where internal states can be aligned thus seems unnecessary.

Let the visible components visible elements of configurations in $D$ be identified by two projection mappings, $\Out \in \pow(D) \rightarrow \pow(D)$ and $\In \in \pow(D) \rightarrow \pow(D)$, identifying its ``inputs'' and ``outputs'', respectively. The set of all possible inputs and outputs in $N$ are given by the two images $\In[\pow(D)] = \{ \In(p) \mid p \in \pow(D) \}$ and $\Out[\pow(D)] = \{ \Out(p) \mid p \in \pow(D) \}$; we denote the set of inputs by $I$. Note that, since models generally cannot control their input signals, transitions from a state $d \in \pow(D)$ are driven by its intersection with a chosen input $i \in I$, i.e. $N(d \cap i)$ yields the next state in $N$.

% a transition for $N$ in response to some input $i \in I$ is given by $N(p \cap i)$, for any initial state $p \in \pow(D)$.

A \textit{trajectory} of $N$ is a non-empty sequence of configurations, $\tau \in D^{+}$, derived from the states generated by $N$ in response to a \textit{driver} $\delta \in I^{+}$ and started in its most general state $\pow(D)$. Specifically, let $\tau_{0} = D$ and $\tau_{n+1} = N(\tau_{n} \cap \delta_{n})$ for all $n \in \mathbb{N} : n < | \delta |$. We denote the trajectory of $N$ induced by a $\delta$ by $N_{\delta}$ and note that it is prefix-closed. \todo{The set of all possible drivers, and the trajectories they induce, describes every behaviour of $N$ for the arbitrary initial state.}

% Because $D$ is finite, we also see that the set of all possible trajectories of $N$ captures every behaviour for starting in the arbitrary initial state.

% The trajectory of $N$ induced by a $\delta$ is denoted by $N_{\delta}$ and is prefix-closed.

With a slight abuse of notation, we extend $\Out(\cdot)$ to sequences component-wise and overload it to accept configurations $D$. \todo{Furthermore}, we assume that both inputs and outputs of $M$ are contained by $N$, i.e. $\In[\pow(C)] \subseteq \In[\pow(D)]$ and $\Out[\pow(C)] \subseteq \Out[\pow(D)]$. \todo{Finally}, we say the circuit $M$ refines the circuit $N$, denoted by $M \leq N$, iff:

\begin{equation*}
\forall \delta \in I^{+} : \Out(M_{\delta}) \subseteq \Out(N_{\delta})
\end{equation*}

\subsection{Lattice-Theoretic refinement}

Text.
