\section{Set-theoretic STE} \label{sec:set-ste}

Consider an arbitrary, but fixed, digital circuit $M$ operating in discrete time. A \textit{configuration} of $M$, denoted by $\sC$, is non-empty and finite set that represents a snapshot of $M$ at a discrete point in time. If the circuit $M$ has $m$ boolean signals, then its set of configurations is typically represented as a sequence $\mathbb{B}^{m}$, where $\mathbb{B} = \{ 0,1 \}$ is the set of boolean values.

\subsubsection{Circuit Model} A simple conceptual model of $M$ is a \textit{transition relation}, $M_{R} \subseteq \sC \times \sC$, where $(c,c') \in M_{R}$ means that $M$ can move from $c$ to $c'$ in one step\footnotemark. The power set of $\sC$, denoted by $\pow(\sC)$, can be viewed as a the set of \textit{predicates} on configurations, where $\cap$, $\cup$, and $\subseteq$ correspond to conjunction, disjunction and implication, respectively. We denote by $\cap S$ and $\cup S$ the intersection and union of all members of any $S \subseteq \pow(\sC)$.

\footnotetext{\todo{Mention how this affects circuits with zero-delays?}}

$M_{R}$ induces a \textit{predicate transformer} $M_{F} \in \pow(\sC) \rightarrow \pow(\sC)$ using the relational image operation:

\begin{equation*}
M_{F}(C) = \{ c' \in \sC \mid \exists c \in C : (c,c') \in M_{R} \}
\end{equation*}

\noindent It is intuitively obvious that if $M$ is in one of the configurations in $C \in \pow(\sC)$, then in one time step it must be in one of the configurations in $M_{F}(p)$. We also see that $M_{F}$ distributes over arbitrary unions:

\begin{equation*}
M_{F}(\cup S) = \cup \{ M_{F}(C) \mid C \in S \}
\end{equation*}

\noindent for all $S \subseteq \pow(\sC)$. In general, any $M_{F}$ that satisfies this distributive property also defines a $M_{R}$ through the equivalence $(c,c') \in M_{R} \iff c' \in M_{F}(\{ c \})$, that is to say, there is no loss of information going from $M_{R}$ to $M_{F}$ or vice versa. We adopt this functional model of $M$ and drop its subscript.

% It follows that $M_{F}$ preserves the empty set of constraints, i.e. $M_{F}(\emptyset) = \emptyset$, and is monotonic, i.e. $p \subseteq q \implies M_{F}(p) \subseteq M_{F}(q)$ for all $p, q \in \pow(C)$.

Exactly what $\sC$ and its signals are, is not important in this section. In practice, however, signals are typically divided into external, such as ``inputs'' and ``outputs'', and internal parts. While an input signal is generally controlled by the external environment, and thus unconstrained by $M$ itself, non-input signals are determined by the circuit topology and functionality. For example, supposed $M$ is the earlier example of a unit-delayed two-input AND gate, we could then define its model $M \in \pow(\mathbb{B}^{3}) \rightarrow \pow(\mathbb{B}^{3})$ as follows:

% although not required here

\begin{equation*}
M(C) = \{ \langle b_{1}, b_{2}, i_{1} \wedge i_{2} \rangle \in \mathbb{B}^{3} \mid \langle i_{1}, i_{2}, o \rangle \in C \}
\end{equation*}

\noindent Here $i_{1}$ and $i_{2}$ refer to the two inputs of the AND gate, $o$ the ignored output, and $b_{1}$ and $b_{2}$ are unconstrained inputs for the new configurations.

\subsubsection{Assertions and satisfaction} \label{sec:set-ste-sat}

A \textit{trajectory assertion} for $M$ is quintuple $A = (S, s_{0}, R, \Antecedent, \Consequent)$, where $S$ is a finite set of \textit{states}, $s_{0} \in S$ is an \textit{initial state}, $R \subseteq S \times S$ is a \textit{transition relation}, $\Antecedent \in S \rightarrow \pow(\sC)$ and $\Consequent \in S \rightarrow \pow(\sC)$ label each state $s$ with an \textit{antecedent} $\Antecedent(s)$ and a \textit{consequent} $\Consequent(s)$. We assume that $(s,s_{0}) \notin S$ for all $s \in S$ without any loss of generality.

The circuit model $M$ intuitively \textit{satisfies} an assertion $A$ if, for every \textit{trajectory} $\tau$ through $M$ and every \textit{run} $\rho$ through $A$, $\tau$ satisfying the antecedents of $\rho$ entails that $\tau$ also satisfies the consequents of $\rho$. To be specific, a \textit{trajectory} of $M$ is a non-empty sequences of configurations, $\tau \in \sC^{+}$, such that $\tau_{n} \in M(\{ \tau_{n-1} \})$ for all $n \in \mathbb{N} : 0 < n < | \tau |$. And a \textit{run} of $A$ is a non-empty sequence of states, $\rho \in S^{+}$, such that $\rho_{0} = s_{0}$ and $(\rho_{n-1}, \rho_{n}) \in R$ for all $n \in \mathbb{N} : 0 < n < | \rho |$. A $\tau$ satisfies the antecedents of $\rho$, denoted by $\tau \aModels \rho$, iff $\tau_{n} \in \Antecedent(\rho_{n})$ for all $n \in \mathbb{N} : n < | \tau | = | \rho |$; satisfaction of consequents is defined similarly with $\Consequent$ and denoted by $\tau \cModels \rho$.


That $M$ satisfies $A$, denoted by $M \models A$, can then be formalized as follows: % \footnotemark

\begin{equation*}
\forall \tau \in \Traj(M) : \forall \rho \in \Runs(A) : | \tau | = | \rho | \implies (\tau \aModels \rho \implies \tau \cModels \rho)
\end{equation*}

\noindent where $\Traj(M)$ and $\Runs(A)$ denote the sets of all trajectories of $M$ and runs of $A$, respectively. \todo{This satisfaction can be formulated equivalently as a problem for deterministic finite automaton.}

%% \footnotetext{\todo{This is equivalent to a DFA formulation~\cite{chou1999}.}}

%% For all functions $\Phi \in S \rightarrow \pow(C)$ and states $s \in S$, define $F \in S \rightarrow (\pow(C) \rightarrow \pow(C))$ as follows:

%% \begin{align}
%% F(s)(p) &= M(\Antecedent(s) \cap p) \\
%% \mathcal{F}(\Phi)(s) &= \Stmt{if } (s = s_{0}) \Stmt{ then } C \Stmt{ else } \cup \{ F(s')(\Phi(s')) \mid (s',s) \in R \}
%% \end{align}

%% \noindent $F$ preserves $\emptyset$, and both $F$ and $\mathcal{F}$ are montonic; two $\Phi, \Phi' \in S \rightarrow \pow(C)$ are ordered as $\Phi \subseteq \Phi' \iff \forall s \in S : \Phi(s) \subseteq \Phi'(s)$. Let $\Phi_{*} \in S \rightarrow \pow(C)$ be the least fixpoint of the equation $\Phi = \mathcal{F}(\Phi)$~\cite{davey2002}. Since both $S$ and $\pow(C)$ are finite, $\Phi_{*}$ is given by $\lim \, \Phi_{n}(s)$, where $\Phi_{n}$ is defined as follows:

%% \begin{equation}
%% \Phi_{n} = \Stmt{if } (n = 0) \Stmt{ then } (\lambda s \in S : \bot) \Stmt{ else } \mathcal{F}(\Phi_{n-1})
%% \end{equation}

%% $M$ \textit{satisfies} a trajectory assertion $A$, denoted by $M \smodels A$, iff $\forall s \in S : \Phi_{*}(s) \cap \Antecedent(s) \subseteq \Consequent(s)$.
