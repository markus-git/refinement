% This is samplepaper.tex, a sample chapter demonstrating the
% LLNCS macro package for Springer Computer Science proceedings;
% Version 2.20 of 2017/10/04
%
\documentclass[runningheads]{llncs}

\usepackage{xcolor} % For coloring todo-text.
\usepackage{amsmath}
\usepackage{amssymb}
\usepackage{stmaryrd} % For double square brackets
\usepackage{graphicx}
% Used for displaying a sample figure. If possible, figure files should
% be included in EPS format.
%
% If you use the hyperref package, please uncomment the following line
% to display URLs in blue roman font according to Springer's eBook style:
% \renewcommand\UrlFont{\color{blue}\rmfamily}

%%%%%%%%%%%%%%%%%%%%%%%%%%%%%%%%%%%%%%%%%%%%%%%%%%%%%%%%%%%%%%%%%%%%%%%%%%%%%%%%
% New commands
%%%%%%%%%%%%%%%%%%%%%%%%%%%%%%%%%%%%%%%%%%%%%%%%%%%%%%%%%%%%%%%%%%%%%%%%%%%%%%%%

%% Editing stuff.
\newcommand{\todo}[1]{\textcolor{red}{#1}}
\newcommand{\Stmt}[1]{\textrm{\textbf{#1}}}

%% Math preferences & missing stuff.
\newcommand{\Alpha}{\mathrm{A}}
\newcommand{\omicron}{\mathrm{o}}
\newcommand{\pow}[1]{\mathcal{P}#1}
\newcommand{\comp}{\mkern -2mu \circ}
\newcommand{\qedb}{\hfill\blacksquare}
\newcommand{\qedw}{\hfill\ensuremath{\Box}}
\renewcommand{\iff}{\Leftrightarrow}
\renewcommand{\implies}{\Rightarrow}
\DeclareMathOperator{\nsqsubseteq}{\not\sqsubseteq}
\DeclareMathOperator{\meet}{\sqcap}
\DeclareMathOperator{\join}{\sqcup}

% STE orig. grammar.
\newcommand{\X}{\mathrm{X}}
\DeclareMathOperator{\Next}{\textbf{N}} % Next-time.
\DeclareMathOperator{\Is}{\mathbin{\textbf{is}}} % comparison.
\DeclareMathOperator{\Traj}{\textit{Traj}} % set of trajectories
\DeclareMathOperator{\Runs}{\textit{Runs}} % set of runs

% Refinement ...
\DeclareMathOperator{\In}{\dot{\iota}}
\DeclareMathOperator{\Out}{\omicron}
%
\DeclareMathOperator{\Inn}{\hat{\iota}}
\DeclareMathOperator{\Outt}{\hat{\omicron}}
%
\DeclareMathOperator{\refine}{\leq}
\DeclareMathOperator{\srefine}{\refine_{\textrm{set}}}
\DeclareMathOperator{\smodels}{\models_{\textrm{set}}}
\DeclareMathOperator{\lrefine}{\refine_{\textrm{lat}}}
\DeclareMathOperator{\lmodels}{\models_{\textrm{lat}}}
%
\DeclareMathOperator{\cc}{\lll} %{\leq} %{\prec\mkern-4mu\prec}

%% Old.
% \DeclareMathOperator{\enc}{\mathrel{\preceq}} % encodes.
% \newcommand{\Driv}[2]{\Traj{#1}(#2)} % trajectory induced by driver.
% \newcommand{\inp}[1]{\llbracket#1\rrbracket} % input mapping
% \newcommand{\out}[1]{\llbracket#1\rrbracket} % output mapping.

%%%%%%%%%%%%%%%%%%%%%%%%%%%%%%%%%%%%%%%%%%%%%%%%%%%%%%%%%%%%%%%%%%%%%%%%%%%%%%%%
% Document
%%%%%%%%%%%%%%%%%%%%%%%%%%%%%%%%%%%%%%%%%%%%%%%%%%%%%%%%%%%%%%%%%%%%%%%%%%%%%%%%

\begin{document}

%%\title{Contribution Title\thanks{Supported by organization x.}}
\title{Refinement for Symbolic Trajectory Evaluation}

%\titlerunning{Abbreviated paper title}
% If the paper title is too long for the running head, you can set
% an abbreviated paper title here

%% \author{First Author\inst{1}\orcidID{0000-1111-2222-3333} \and
%% Second Author\inst{2,3}\orcidID{1111-2222-3333-4444} \and
%% Third Author\inst{3}\orcidID{2222--3333-4444-5555}}
\author{Authors}

\authorrunning{Authors}
% First names are abbreviated in the running head.
% If there are more than two authors, 'et al.' is used.
%
%% \institute{Princeton University, Princeton NJ 08544, USA \and
%% Springer Heidelberg, Tiergartenstr. 17, 69121 Heidelberg, Germany
%% \email{lncs@springer.com}\\
%% \url{http://www.springer.com/gp/computer-science/lncs} \and
%% ABC Institute, Rupert-Karls-University Heidelberg, Heidelberg, Germany\\
%% \email{\{abc,lncs\}@uni-heidelberg.de}}
\institute{Chalmers}

\maketitle % typeset the header of the contribution

\begin{abstract}
Model refinement such that it preserves symbolic trajectory evalutions.

\keywords{STE \and Refinement \and ?}
\end{abstract}

\section{Introduction to STE}

\subsection{Original STE}

\textit{Symbolic trajectory evaluation}~\cite{seger1995} (STE) is a high-performance model checking technique based on \textit{symbolic simulation} extended with a temporal \textit{next-time} operator to describe circuit behaviour over time. In its simplest form, STE tests the validity of an \textit{assertion} of the form $A \Rightarrow C$, where both the \textit{antecedent} $A$ and \textit{consequent} $C$ are formulas in the following logic:

\begin{equation*}
f ::= p \: | \: f \wedge f \: | \: P \rightarrow f \: | \: \Next f
\end{equation*}

\noindent Here, $p$ is a simple predicate over ``values'' in a circuit and $P$ is a Boolean propositional formula, and the operators $\wedge$, $\rightarrow$ and $\Next$ are conjunction, domain restriction and the next-time operator, respectively.

If the circuit contains Boolean signals, $p$ is typically drawn from the following two predicates: $n \Is 1$ and $n \Is 0$, where $n$ ranges over the signals (or nodes) in a circuit. For example, suppose we have a unit-delayed, two-input AND-gate, then it is reasonable to assume that the assertion $(\mathit{in_{1}} \Is 1 \wedge \mathit{in_{2}} \Is 1) \Rightarrow \Next (\mathit{out} \Is 1)$ is true. Indeed, STE efficiently validates such statements for us.

While the truth semantics of an assertion in STE is defined as the satisfaction of its ``defining'' trajectory (bounded sequence of states) relative to a model structure of the circuit, what the STE algorithm computes is exactly the solution of a data-flow equation~\cite{chou1999} in the classic format~\cite{muchnick1997}. \dots

% translate an STE assertion into a linear, directed graph \dots introducing as many states as the assertion is deep \dots labelling each state with the antecedent and consequent at that depth. \dots

% We adopt this reimagined STE as data flow analysis~\cite{chou1999}, in which trajectory assertions can have arbitrary state-transition graphs. The following sections give a short introduction to its mathematical foundation.

\subsection{Lattice-theoretic STE}

Consider an arbitrary, but fixed, digital circuit $M$ operating in discrete time. A \textit{configuration} of $M$, denoted by $C$, is non-empty and finite set that represents a snapshot of $M$ at a discrete point in time. If the circuit $M$ has $m$ boolean signals, then its set of configurations is typically represented as a sequence $\mathbb{B}^{m}$, where $\mathbb{B} = \{ 0,1 \}$ is the set of boolean values.

\subsubsection{Circuit Model} A simple conceptual model of $M$ is a \textit{transition relation}, $M_{R} \subseteq C \times C$, where $(c,c') \in M_{R}$ means that $M$ can move from $c$ to $c'$ in one step\footnotemark. The power set of $C$, denoted by $\mathcal{P}(C)$, can be viewed as a the set of \textit{predicates} on configurations, where $\cap$, $\cup$, and $\subseteq$ correspond to conjunction, disjunction and implication, respectively. Furthermore, for any $Q \subseteq \mathcal{P}(C)$, we denote by $\cap Q$ and $\cup Q$ the intersection and union of all members of $Q$.

\footnotetext{\todo{Mention how this affects circuits with zero-delays?}}

$M_{R}$ induces a \textit{predicate transformer} $M_{F} \in \mathcal{P}(C) \rightarrow \mathcal{P}(C)$ using the relational image operation:

\begin{equation*}
M_{F}(p) = \{ c' \in C \mid \exists c \in p : (c,c') \in M_{R} \}
\end{equation*}

\noindent It is intuitively obvious that if $M$ is in one of the configurations in $p \in \mathcal{P}(C)$, then in one time step it must be in one of the configurations in $M_{F}(p)$. Furthermore, from its definition we see that $M_{F}$ distributes over arbitrary unions:

\begin{equation*}
M_{F}(\cup Q) = \cup \{ M_{F}(q) \mid q \in Q \}
\end{equation*}

\noindent for all $Q \subseteq \mathcal{P}(C)$. Any $M_{F}$ that satisfies this distributive property also defines a $M_{R}$ through the equivalence $(c,c') \in M_{R} \iff c' \in M_{F}(\{ c \})$, that is to say, there is no loss of information going from $M_{R}$ to $M_{F}$ or vice versa. We adopt this functional model of $M$ and drop its subscript. It follows its distributivity that $M$ also preserves the empty set of constraints, i.e. $M(\emptyset) = \emptyset$, and that $M$ is monotonic, i.e. $p \subseteq q \implies M(p) \subseteq M(q)$ for all $p, q \in \mathcal{P}(C)$.

In practice, signals in $M$ are typically divided into ``input'' signals and ``output'' or ``internal'' signals. While an input signal is typically controlled by the external environment, and thus unconstrained by $M$ itself, non-input signals are determined by the circuit topology and functionality. For example, supposed $M$ is the earlier example of a unit-delayed two-input AND gate, we could then define its model $\mathcal{M} \in \mathcal{P}(\mathbb{B}^{3}) \rightarrow \mathcal{P}(\mathbb{B}^{3})$ as:

% although not required here

\begin{equation*}
\mathcal{M}(p) = \{ \langle b_{1}, b_{2}, i_{1} \wedge i_{2} \rangle \in \mathbb{B}^{3} \mid \langle i_{1}, i_{2}, o \rangle \in p \}
\end{equation*}

\noindent Here $i_{1}$ and $i_{2}$ refer to the two inputs of the AND gate and $o$ the ignored output; $b_{1}$ and $b_{2}$ are unconstrained inputs in the new configuration.

\subsubsection{Ternary lattices}

Manipulating subsets of $\mathbb{B}^{m}$ is however impractical for even moderately large $m$, which leads us to one of the key insights of STE. Namely, instead of manipulating subsets of $\mathbb{B}^{m}$ directly, one can use sequences of ternary values $\mathbb{T} = \mathbb{B} \cup \{ \X \} $ to approximate them, whose sizes are only linear in $m$. Here the $1$ and $0$ from $\mathbb{B}$ denotes specific, defined values whereas $\X$ denotes an ``unknown'' value that could be either $1$ or $0$. This intuition induces a partial order $\sqsubseteq$ on $\mathbb{T}$, where $0 \sqsubseteq \X$ and $1 \sqsubseteq \X$\footnotemark. For any $m \in \mathbb{N}$, this ordering on $\mathbb{T}$ is lifted component-wise to $\mathbb{T}^{m}$.

% As an example, we have that $\langle 1,1,0 \rangle$ and $\langle 1,0,0 \rangle$ are both $\sqsubseteq \langle 1,X,0 \rangle \in \mathbb{T}^{3}$ because they all agree on their first and third element and $X$ can be both $0$ and $1$.

\footnotetext{We use the reverse ordering of what is originally used in STE.}

Note that $\mathbb{T}^{m}$ does not quite form a complete lattice because it lacks a bottom: both $0 \sqsubseteq \X$ and $1 \sqsubseteq \X$ but $0$ and $1$ are equally defined. A special bottom element $\bot$ is therefore introduced, such that $\bot \sqsubseteq t$ and $\bot \neq t$ for all $t \in \mathbb{T}^{m}$. The extended $\mathbb{T}_{\bot}^{m} = \mathbb{T}^{m} \cup \{ \bot \}$ then becomes a complete lattice. We denote the top element $\langle \X, \dots, \X \rangle$ of $\mathbb{T}_{\bot}^{m}$ by $\top$.

Generalising from any specific domain, let $(\hat P,\sqsubseteq)$ be a finite, complete lattice of \textit{abstract predicates} in which the meet $\sqcap$ and join $\sqcup$ of any subset $Q \subseteq \hat P$ exists. Similar to the previous set operations for power sets, $\sqcap$, $\sqcup$ and $\sqsubseteq$ correspond to conjunction, disjunction and implication for abstract predicates, respectively. Furthermore, for any $Q \subseteq \hat P$, we denote by $\sqcap Q$ and $\sqcup Q$ the meet and join of all members of $Q$.

\subsubsection{Abstract circuit model}

Let there be a Galois connection relating ``concrete'' predicates $\mathcal{P}(C)$ and abstract predicates $\hat P$. The usual definition of a Galois connection is in terms of an \textit{abstraction} $\alpha \in \mathcal{P}(C) \rightarrow \hat P$ and a \textit{concretisation} $\gamma \in \hat P \rightarrow \mathcal{P}(C)$ function, such that $\alpha(p) \sqsubseteq \hat p \iff p \subseteq \gamma(\hat p)$ for all $p \in \mathcal{P}(C)$ and $\hat p \in \hat P$. For example, a Galois connection from $\mathcal{P}(\mathbb{B}^{m})$ to $\mathbb{T}_{\bot}^{m}$ for any $m \in \mathbb{N}$ can be defined in a natural way by its concretisation function $\Gamma \in \mathbb{T}_{\bot}^{m} \rightarrow \mathcal{P}(\mathbb{B}^{m})$:

\begin{align*}
\Gamma ( \langle t_{0}, \dots,t_{m-1} \rangle ) &= \{ \langle b_{0}, \dots,b_{m-1} \rangle \in \mathbb{B}^{m} \mid \forall i < m : t_{i} \neq \X \Rightarrow b_{i} = t_{i} \} \\
\Gamma ( \bot ) &= \emptyset
\end{align*}

\noindent Listing each concrete predicate approximated by $\langle 1, \X, 0 \rangle$ with $\Gamma$, we get the set $\{ \langle 1, 1, 0 \rangle,\allowbreak \langle 1, 0, 0 \rangle \}$ as desired. Conversely, we can define the abstraction function $\Alpha \in \mathcal{P}(\mathbb{B}^{m}) \rightarrow \mathbb{T}_{\bot}^{m}$ by finding the most precise abstraction:

\begin{align*}
\Alpha ( p ) &= \sqcup \{ \langle t_{0}, \ldots, t_{m-1} \rangle \in \mathbb{T}_{\bot}^{m} \mid \langle b_{0}, \ldots, b_{m-1} \rangle \in p, \forall i < m : t_{i} = b_{i} \} \\
\Alpha ( \emptyset ) &= \bot
\end{align*}

\noindent Abstracting the concrete predicates $\{ \langle 1, 1, 0 \rangle, \langle 1, 0, 0 \rangle \}$ with $\Alpha$ indeed yields the original $\langle 1, \X, 0 \rangle$ abstraction.

We view a Galois connection from $\mathcal{P}(C)$ to $\hat P$ instead as binary relation, and define $\ll \, \subseteq \mathcal{P}(C) \times \hat P$, such that for all $Q \subseteq \mathcal{P}(C)$ and $\hat Q \subseteq \hat P$:

\begin{equation*}
\forall p \in Q : \forall \hat p \in \hat Q : p \ll \hat p \iff \cup Q \ll \sqcap \hat Q
\end{equation*}

\noindent Where $p \ll \hat p$ reads as ``$p$ can be approximated as $\hat p$''\footnotemark. The standard abstraction and concretisation functions can be derived from $\ll$ as $\gamma(\hat p) \iff \cup \{ p \in \mathcal{P}(C) \mid p \ll \hat p \}$, and vice versa as $p \ll \hat p \iff p \subseteq \gamma(\hat p)$; similar derivations can be made to and from $\alpha$. Intuitively, $\ll$ is an extension of the partial order $\subseteq$ of $\mathcal{P}(C)$ and $\sqsubseteq$ of $\hat P$ to an ordering between $\mathcal{P}(C)$ and $\hat P$.

\footnotetext{\todo{Used to prove thesis in~\cite{chou1999}, included here since we use one for refinment later on.}}

An \textit{abstract predicate transformer} $\hat M \in \hat P \rightarrow \hat P$ is an \textit{abstract interpretation}~\cite{cousot1996} of $M \in \mathcal{P}(C) \rightarrow \mathcal{P}(C)$ iff: (1) $\hat M$ preserves $\bot$, i.e. $\hat M(\bot) = \bot$; (2) $\hat M$ is monotonic, i.e. $\hat p \sqsubseteq \hat q \Rightarrow \hat M (\hat p) \sqsubseteq \hat M (\hat q)$ for all $\hat p, \hat q \in \hat P$; and (3) $\ll$ is a \textit{simulation relation} from $\mathcal{P}(C)$ to $\hat P$, i.e. $p \ll \hat p \Rightarrow M (p) \ll \hat M (\hat p)$ for all $p \in \mathcal{P}(C)$ and $\hat p \in \hat P$. That $\ll$ is a simulation relation can also be stated in terms of its abstraction $\alpha$ and concretisation $\gamma$ functions: $\alpha(M(p)) \sqsubseteq \hat M(\alpha(p))$ for all $p \in \mathcal{P}(C)$, and $M(\gamma(\hat p)) \subseteq \gamma(\hat M(\hat p))$ for all $\hat p \in \hat P$.

Note that $\hat M$ does not distribute over arbitrary join in general because information is potentially discarded when joining two lattices. As an example, we define $\hat{\mathcal{M}}$ as an abstract interpretation for the earlier $\mathcal{M}$:

% , that would have been kept by union in the original model $M$

\begin{align*}
\hat{\mathcal{M}}(\langle p_{1}, p_{2}, p_{3} \rangle) &=
\left\{
  \begin{array}{ll}
    \langle \X, \X, 1 \rangle, & \textrm{if } p_{1} = 1 \textrm{ and } p_{2} = 1 \\
    \langle \X, \X, 0 \rangle, & \textrm{if } p_{1} = 0 \textrm{ or } p_{2} = 0 \\
    \langle \X, \X, \X \rangle, & \textrm{otherwise}
  \end{array}
\right. \\
\hat{\mathcal{M}}(\bot) &= \bot
\end{align*}

% Note: Can we shorten the def. of funny M somehow? Or use a simpler gate.

\noindent If we apply $\hat{\mathcal{M}}$ to the join of $\langle 0, 1, \X \rangle$ and $\langle 1, 0, \X \rangle$, or if we apply $\hat{\mathcal{M}}$ to them individually and then join the results, we get two different results:

\begin{equation*}
\begin{array}{lll}
  \hat{\mathcal{M}}(\langle 0, 1, \X \rangle \sqcup \langle 1, 0, \X \rangle) &= \hat{\mathcal{M}}(\langle \X, \X, \X \rangle) &= \langle \X, \X, \X \rangle \\
  \hat{\mathcal{M}}(\langle 0, 1, \X \rangle) \sqcup \hat{\mathcal{M}}(\langle 1, 0, \X \rangle) &= \langle \X, \X, 0 \rangle \sqcup \langle \X, \X, 0 \rangle &= \langle \X, \X, 0 \rangle
\end{array}
\end{equation*}

\noindent The inequality $\sqcup \{ \hat M(\hat q) \mid \hat q \in \hat Q \} \sqsubseteq \hat M(\sqcup \hat Q)$ for all $\hat Q \sqsubseteq \hat P$ does however hold, since it is implied by the monotonicity of $\hat M$.

\subsubsection{Assertions and satisfaction}

A \textit{trajectory assertion} for $\hat M$ is a quintuple $\hat A = (S,s_{0},R,\pi_{a},\pi_{c})$, where $S$ is a finite set of \textit{states}, $s_{0} \in S$ is an \textit{initial state}, $R \subseteq S \times S$ is a \textit{transition relation}, $\pi_{a} \in S \rightarrow \hat P$ and $\pi_{c} \in S \rightarrow \hat P$ label each state $s$ with an \textit{antecedent} $\pi_{a}(s)$ and a \textit{consequent} $\pi_{c}(s)$. Furthermore, we assume that $(s,s_{0}) \notin S$ for all $s \in S$ without any loss of generality.

For all $\Phi \in S \rightarrow \hat P$ and $s \in S$, define $F \in S \rightarrow (\hat P \rightarrow \hat P)$ and $\mathcal{F} \in (S \rightarrow \hat P) \rightarrow (S \rightarrow \hat P)$ as follows:

\begin{align}
F(s)(\hat p) &= \hat M(\pi_{a}(s) \sqcap \hat p) \\
\mathcal{F}(\Phi)(s) &= \Stmt{if } (s = s_{0}) \Stmt{ then } \top \Stmt{ else } \sqcup \{ F(s')(\Phi(s')) \mid (s',s) \in R \}
\end{align}

\noindent $F$ preserves $\bot$ and both $F$ and $\mathcal{F}$ are monotonic, where two $\Phi, \Phi' \in S \rightarrow \hat P$ are ordered as $\Phi \sqsubseteq \Phi' \iff \forall s \in S : \Phi(s) \sqsubseteq \Phi'(s)$. Let $\Phi_{*} \in S \rightarrow \hat P$ be the least fixpoint of the equation $\Phi = \mathcal{F}(\Phi)$~\cite{davey2002}. Since both $S$ and $\hat P$ are finite, $\Phi_{*}$ is given by $\lim \, \Phi_{n}(s)$ where $\Phi_{n}$ is defined as follows:

\begin{equation}
\Phi_{n} = \Stmt{if } (n = 0) \Stmt{ then } (\lambda s \in S : \bot) \Stmt{ else } \mathcal{F}(\Phi_{n-1})
\end{equation}

We say that the abstract circuit $\hat M$ \textit{satisfies} a lattice-based, abstract trajectory assertion $\hat A$, denoted by $\hat M \models \hat A$, iff:

\begin{equation}
\forall s \in S : \Phi_{*}(s) \sqcap \pi_{\alpha}(s) \sqsubseteq \pi_{c}(s)
\end{equation}

\noindent $\hat M \models \hat A$ implies that a concretisation of $\hat A$ can also be satisfied by the original, set-based model $M$ \cite{chou1999}.

\section{System refinement (WIP)}

Consider another fixed, but arbitrary, circuit $N$ such that configurations of $M$ and $N$ have the same number of externally visible elements but can differ internally. Let $\hat N \in \hat Q \rightarrow \hat Q$ be an abstract predicate transformer of $N$, we then we say that $\hat M$ \textit{refines} $\hat N$ if every \textit{externally visible behaviour} allowed by $\hat M$ is also allowed by $\hat N$, regardless of any initial configurations.

Let the \textit{externally visible} parts of an abstract predicate $\hat P$ be the subsets given by two projections, $\I$ and $\O$, identifying the ``inputs'' and ``outputs'' of $\hat P$, respectively. Further, let $\out{\cdot} \in \hat P \rightarrow \hat O$ be a mapping that takes each $\hat p \in \hat P$ to its visible outputs $\out{\hat p} \in \hat O$; $\out{\cdot}$ is extended to sequences component-wise. With a slight abuse of notation, we overload both projections and the mapping to also accept predicates from $\hat Q$ and note that $\I(\hat P) = \I(\hat Q) = \hat I$ and $\O(\hat P) = \O(\hat Q) = \hat O$.

% , the kind used will be apparent from its context.

A \textit{driver} of $\hat M$ and $\hat N$ is a nonempty sequence of inputs, $\delta \in \hat I^{+}$, and induces a trajectory $\tau$ in $\hat M$ (resp. $\hat N$) where $\tau[0] = \top$ and $\forall i \in \mathbb{N} : 0 < i < | \delta + 1 | \implies \tau[j] = \hat M(\tau[j-1] \sqcap \delta[j-1])$. The set of drivers is denoted by $\Driv$, and the trajectory induced by a driver $\delta$ in $\hat M$ is denoted by $\Traj(\hat M)(\delta)$. Finally, we say that $\hat M$ refines $\hat N$, denoted by $\hat M \approx \hat N$, iff:

\begin{equation*}
\forall \delta \in \hat I^{+} : \out{\Traj(\hat M)(\delta)} \sqsubseteq \out{\Traj(\hat N)(\delta)}
\end{equation*}

\noindent Intuitively, if every possible input sequence produces the same, or at least more specified, outputs, then every visible behaviour of $\hat M$ in covered by $\hat N$.

Let $\enc \in \hat P \times \hat Q$ (``encodes'') be a simulation relation such that $\hat p \enc \hat q$ implies that (1) $\out{\hat p} \sqsubseteq \out{\hat q}$ and (2) $\hat M(\hat p \sqcap \hat \iota) \enc \hat N(\hat q \sqcap \hat \iota)$ for all inputs $\hat \iota \in \hat I$. The relation is extended to transformers such that $\hat M \enc \hat N$ iff the top element of $\hat P$ encodes that of $\hat Q$, $\top \enc \top$. We then simplify refinement thus: $\hat M \enc \hat N \iff \hat M \approx \hat N$.

% the top element of $\hat P$ encodes that of $\hat Q$, $\top(\hat P) \enc \top(\hat Q)$

% $\hat \iota \enc \hat \iota$ for all inputs $\hat \iota \in \hat I$

% That is to say, the outputs of $\hat M$ is as specified as those of $\hat N$ for any input sequence starting in the state where nothing else is assumed.

% If there exists a $\hat q \in \hat Q$ for every $\hat p \in \hat P$ such that $\hat p \preceq \hat q$, then $M$ must ``cover'' the input-output behaviour of $N$.

% LocalWords:  consequents


\appendix
\section{Proofs}

\subsection{Lemma: $\hat M \enc \hat N \iff \hat M \leq \hat N$}

Recall that $\hat p \enc \hat q$ implies (1) $\out{\hat p} \sqsubseteq \out{\hat q}$ and (2) $\forall \hat \iota \in \hat I : \hat M(\hat \iota \sqcap \hat p) \enc \hat N(\hat \iota \sqcap \hat q)$. Further $\hat M \enc \hat N$ implies that $(\top \in \hat P) \enc \, (\top \in \hat Q)$. The trajectory $\tau$ induced by a driver $\delta$ is defined as follows: $\tau[0] = \top$ and $\forall i \in \mathbb{N} : i < | \delta | \implies \tau[i+1] = \hat M(\delta[i] \sqcap \tau[i])$. That $\hat M \leq \hat N$ iff:

\begin{equation*}
\forall \delta \in \hat I^{+} : \out{\Driv{\hat M}{\delta}} \sqsubseteq \out{\Driv{\hat N}{\delta}}
\end{equation*}

We prove the two directions of $\hat M \enc \hat N \iff \hat M \leq \hat N$ separately.

\subsubsection{$\hat M \enc \hat N \implies \hat M \leq \hat N$:} For any $\delta \in \hat I^{+}$, let $\tau = \langle \hat p_{0}, \hat p_{1}, \ldots \rangle \in \hat P^{+}$ and $v = \langle \hat q_{0}, \hat q_{1}, \dots \rangle \in \hat Q^{+}$ be the induced trajectories $\Driv{\hat M}{\delta}$ and $\Driv{\hat N}{\delta}$, respectively. We prove that $p_{n} \enc q_{n}$, and thus $\out{p_{n}} \sqsubseteq \out{q_{n}}$, for every $n \in \mathbb{N} : n < | \delta + 1 |$ and $\delta$ by induction on $n$. For the base case, we have $p_{0} = \bot \in \hat P$ and $q_{0} = \bot \in \hat Q$. That $p_{0} \enc q_{0}$, and thus $\out{\hat p_{0}} \sqsubseteq \out{\hat q_{0}}$, follows immediately from the definition of $\hat M \enc \hat N$. For the inductive step, assume that $\hat p_{n} \enc \hat q_{n}$. Following the definition of $\tau$ and $v$, we know that $\hat p_{n+1} = \hat M(\hat \iota \sqcap \hat p_{n})$ and $\hat q_{n+1} = \hat N(\hat \iota \sqcap \hat q_{n})$ for some $\hat \iota \in \hat I$. But property (2) of $\hat p_{n} \enc \hat q_{n}$ states that $\hat M(\hat \iota \sqcap \hat p_{n}) \enc \hat N(\hat \iota \sqcap \hat q_{n})$ for any $\hat \iota \in \hat I$, so we must have that $\hat p_{n+1} \enc \hat q_{n+1}$ and thus $\out{\hat p_{n+1}} \sqsubseteq \out{\hat q_{n+1}}$.

\subsubsection{$\hat M \enc \hat N \Leftarrow \hat M \leq \hat N$:} Define $\enc \in \hat P \times \hat Q$ as follows:

\begin{equation*}
\bigcup_{\delta \in I^{+}} \{ (\Driv{\hat M}{\delta}[i], \Driv{\hat N}{\delta}[i]) \mid i \in \mathbb{N}, i < | \delta + 1 | \}
\end{equation*}

\noindent Here $\Driv{\hat M}{\delta}[i]$ is the i-th predicate of the trajectory induced by a driver $\delta$, that is, for any $\hat p \enc \hat q$, we must have that $\hat p$ and $\hat q$ are a pair of i-th predicates induced by a common $\delta$. By definition $\hat M \leq \hat N$ then, we must have that $\out{\hat p} \sqsubseteq \out{\hat q}$ for any pair $\hat p \enc \hat q$. Furthermore, as $\delta \in \hat I^{+}$, then so must $\delta^{\frown}\hat \iota \in \hat I^{+}$ ($\delta$ followed by $\hat \iota$) for any $\hat \iota \in \hat I$. This definition of $\enc$ thus satisfies both of its desired properties. Finally, that $\hat M \enc \hat N$ follows from how $\out{\top \in \hat P} \sqsubseteq \out{\top \in \hat Q}$ is obviously true and $\hat M(\hat \iota \sqcap \top) \enc \hat N(\hat \iota \sqcap \top)$ for all $\hat \iota \in \hat I$ because $\langle \hat \iota \rangle \in \hat I^{+}$.

\subsection{$\hat M \enc \hat N \implies \forall s \in S : \Phi(s) \enc \Psi(s)$}

Recall that $\hat p \enc \hat q$ implies (1) $\out{\hat p} \sqsubseteq \out{\hat q}$ and (2) $\forall \hat \iota \in \hat I : \hat M(\hat \iota \sqcap \hat p) \enc \hat N(\hat \iota \sqcap \hat q)$. Further $\hat M \enc \hat N$ implies that $(\top \in \hat P) \enc \, (\top \in \hat Q)$.

\begin{align*}
F(s)(\hat p) &= \hat M(\pi_{a}(s) \sqcap \hat p) \\
\mathcal{F}(\Phi)(s) &= \Stmt{if } (s = s_{0}) \Stmt{ then } \top \Stmt{ else } \sqcup \{ F(s')(\Phi(s')) \mid (s',s) \in R \} \\
\Phi_{n} &= \Stmt{if } (n = 0) \Stmt{ then } (\lambda s \in S : \bot) \Stmt{ else } \mathcal{F}(\Phi_{n-1}) \\
\hat M \models \hat A &\implies \forall s \in S : \Phi_{*}(s) \sqcap \pi_{\alpha}(s) \sqsubseteq \pi_{c}(s)
\end{align*}

We first prove a few intermediate lemmas.

\textbf{Lemma (B)}: $(\bot \in \hat P) \enc \, (\bot \in \hat Q)$. Prop. (1) follows immediately as $\out{\bot \in \hat P} = \bot \in \hat O = \out{\bot \in \hat Q}$. Because $\hat M$ and $\hat N$ both preserve bottom, we have that $\hat M(\hat \iota \sqcap \bot) = \bot \in \hat P$ and $\hat N(\hat \iota \sqcap \bot) = \bot \in \hat Q$ for all $\hat \iota \in \hat I$. That is, any path that gets to $\bot$ must stay there, regardless of inputs. Prop. (2) then follows as well.

\textbf{Lemma (FG)}: $\hat p \enc \hat q \implies \forall s \in S : F(s)(\hat p) \enc G(s)(\hat q)$. First, we note that $F(s)(\hat p) = \hat M(\pi_{a}(s) \sqcap \hat p)$ and that $G(s)(\hat q) = \hat N(\pi_{a}(s) \sqcap \hat q)$ for some $\pi_{a}(s) \in \hat I$. That $\hat M(\pi_{a}(s) \sqcap \hat p) \enc \hat N(\pi_{a}(s) \sqcap \hat q)$ then follows directly from prop. (2) of $\hat p \enc \hat q$.

\textbf{Lemma (M)}: $(\hat p \enc \hat q) \wedge (\hat p' \enc \hat q') \implies (\hat p \sqcup \hat p') \enc \, (\hat q \sqcup \hat q')$. Let $\tau = \hat p_{0}, \hat p_{1}, \ldots$ be the trajectory starting from $\hat p_{0} = \hat p$ and driven by $\delta = \langle \hat \iota_{0}, \hat \iota_{1}, \ldots \rangle \in \hat I^{*}$, that is, $\hat p_{n+1} = \hat M(\hat \iota_{n} \sqcap \hat p_{n})$ for all $n \in \mathcal{N} : n < | \delta |$. Similarly, let $\tau'$, $v$ and $v'$ be trajectories driven by the same $\delta$ but starting in $\hat p'$, $\hat q$ and $\hat q'$, respectively. We show by induction on $n$ that, if $\tau \sqsubseteq v$ and $\tau' \sqsubseteq v'$, the trajectories starting from $\hat p \sqcup \hat p'$ and $\hat q \sqcup \hat q'$ satisfy output inequality of $\enc$ for all $\delta$, and thus also that $(\hat p \sqcup \hat p') \enc \, (\hat q \sqcup \hat q')$. For the base case, we have that $\hat p_{0} \enc \hat q_{0}$ and $\hat p_{0}' \enc \hat q_{0}$. By prop. (1) of $\enc$, we also have that $\out{\hat p_{0}} \sqsubseteq \out{\hat q_{0}}$ and $\out{\hat p_{1}} \sqsubseteq \out{\hat q_{1}}$. By definition of $\sqcup$ then, we must have that $\out{\hat p_{0}} \sqsubseteq \out{\hat q_{0} \sqcup \hat q_{0}'}$ and therefore also that $\out{\hat p_{0} \sqcup \hat p_{0}'} \sqsubseteq \out{\hat q_{0} \sqcup \hat q_{0}'}$. For the inductive step, we have that $\hat M(\hat \iota_{n} \sqcap \hat p_{n}) \enc \hat N(\hat \iota_{n} \sqcap \hat q_{n})$ and $\hat M(\hat \iota_{n} \sqcap \hat p_{n}') \enc \hat N(\hat \iota_{n} \sqcap \hat q_{n}')$, and thus by prop. (1) of $\enc$ also that $\out{\hat M(\hat \iota_{n} \sqcap \hat p_{n})} \sqsubseteq \out{\hat N(\hat \iota_{n} \sqcap \hat q_{n})}$ and $\out{\hat M(\hat \iota_{n} \sqcap \hat p_{n}')} \sqsubseteq \out{\hat N(\hat \iota_{n} \sqcap \hat q_{n}')}$. Because $\hat N$ is monotonic, and by the definitions of $\sqcap$ and $\sqcup$, we must have that $\out{\hat N(\hat \iota \sqcap \hat q_{n})} \sqsubseteq \out{\hat N(\hat \iota \sqcap (\hat q_{n} \sqcup \hat q_{n}'))}$ and thus $\out{\hat M(\hat \iota_{n} \sqcap \hat p_{n})} \sqsubseteq \out{\hat N(\hat \iota \sqcap (\hat q_{n} \sqcup \hat q_{n}'))}$. Similarly, because $\hat M$ is monotonic as well, and by definition of $\sqcap$ and $\sqcup$, we have that $\out{\hat M(\hat \iota_{n} \sqcap \hat p_{n})} \sqsubseteq \out{\hat N(\hat \iota \sqcap (\hat p_{n} \sqcup \hat p_{n}'))}$ and the least such bound. Combining these two results, we must have that $\out{\hat M(\hat \iota_{n} \sqcap (\hat p_{n} \sqcup \hat p_{n}'))} \sqsubseteq \out{\hat M(\hat \iota_{n} \sqcap (\hat q_{n} \sqcup \hat q_{n}'))}$.

\textbf{Lemma (UM)}: $R \subseteq S$, $\forall s \in S : \hat p(s) \enc \hat q(s) \implies \sqcup \{ F(s)(\hat p(s)) \mid s \in \: R \} \enc \sqcup \{ G(s)(\hat q(s)) \mid s \in \: R \}$. Pair up the two sets, indexed by $s$. Apply lemma FG to each pair to see that they are encoded and then fold all pairs using lemma M to see that the ``big''-meet is also encoded.

Since $\Phi_{*}(s) = \lim \, \Phi_{n}(s)$ and $\Psi_{*}(s) = \lim \, \Psi_{n}(s)$, it suffices to prove that $\Phi_{n}(s) \enc \Psi_{n}(s)$ for all $s \in S$ and $n \in \mathbb{N}$. We do so by induction on $n$. The base case, where $\Phi_{0}(s) = \bot \in \hat P$ and $\Psi_{0}(s) = \bot \in \hat Q$, follows from lemma X. For the inductive step, assme that $\Phi_{n}(s) \enc \Psi_{n}(s)$ for all $s \in S$. For $s = s_{0}$, we have that $\Phi_{n+1}(s_{0}) = \top \in \hat P$ and $\Psi_{n+1}(s_{0}) = \top \in \hat Q$, which follows from how $\hat M \enc \hat N$ implies that $(\top \in \hat P) \enc \, (\top \in \hat Q)$. For any $s \neq s_{0}$, we have that $\Phi_{n+1}(s) = \mathcal{F}(\Phi_{n})(s) = \sqcup \{ F(s')(\Phi_{n}(s')) \mid (s',s) \in R \} \enc \sqcup \{ F(s')(\Psi_{n}(s')) \mid (s',s) \in R \} \enc \sqcup \{ G(s')(\Psi_{n}(s')) \mid (s',s) \in R \} = \mathcal{G}(\Psi_{n})(s) = \Psi_{n+1}(s)$.

Given that $\Phi_{*}(s) \enc \Psi_{*}(s)$, and thus $\out{\Phi_{*}(s)} \sqsubseteq \out{\Psi_{*}(s)}$, for all $s \in S$, it follows that $\Phi_{*}(s) \sqcap \pi_{a}(s) \sqsubseteq \pi_{c}(s) \implies \Psi_{*}(s) \sqcap \pi_{a}(s) \sqsubseteq \pi_{c}(s)$.



%% \section{First Section}
%% \subsection{A Subsection Sample}
%% Please note that the first paragraph of a section or subsection is
%% not indented. The first paragraph that follows a table, figure,
%% equation etc. does not need an indent, either.
%
%% Subsequent paragraphs, however, are indented.
%
%% \subsubsection{Sample Heading (Third Level)} Only two levels of
%% headings should be numbered. Lower level headings remain unnumbered;
%% they are formatted as run-in headings.
%
%% \begin{equation}
%% x + y = z
%% \end{equation}
%
%% \begin{theorem}
%% This is a sample theorem. The run-in heading is set in bold, while
%% the following text appears in italics. Definitions, lemmas,
%% propositions, and corollaries are styled the same way.
%% \end{theorem}
%
%% \begin{proof}
%% Proofs, examples, and remarks have the initial word in italics,
%% while the following text appears in normal font.
%% \end{proof}
%
% the environments 'definition', 'lemma', 'proposition', 'corollary',
% 'remark', and 'example' are defined in the LLNCS documentclass as well.
%
%% For citations of references, we prefer the use of square brackets
%% and consecutive numbers. Citations using labels or the author/year
%% convention are also acceptable. The following bibliography provides
%% a sample reference list with entries for journal
%% articles~\cite{ref_article1}, an LNCS chapter~\cite{ref_lncs1}, a
%% book~\cite{ref_book1}, proceedings without editors~\cite{ref_proc1},
%% and a homepage~\cite{ref_url1}. Multiple citations are grouped
%% \cite{ref_article1,ref_lncs1,ref_book1},
%% \cite{ref_article1,ref_book1,ref_proc1,ref_url1}.
%
% ---- Bibliography ----
%
% BibTeX users should specify bibliography style 'splncs04'.
% References will then be sorted and formatted in the correct style.
%
\bibliographystyle{splncs04}
\bibliography{lib}
%% \begin{thebibliography}{8}
%% \bibitem{ref_article1}
%% Author, F.: Article title. Journal \textbf{2}(5), 99--110 (2016)
%% \bibitem{ref_lncs1}
%% Author, F., Author, S.: Title of a proceedings paper. In: Editor,
%% F., Editor, S. (eds.) CONFERENCE 2016, LNCS, vol. 9999, pp. 1--13.
%% Springer, Heidelberg (2016). \doi{10.10007/1234567890}
%% \bibitem{ref_book1}
%% Author, F., Author, S., Author, T.: Book title. 2nd edn. Publisher,
%% Location (1999)
%% \bibitem{ref_proc1}
%% Author, A.-B.: Contribution title. In: 9th International Proceedings
%% on Proceedings, pp. 1--2. Publisher, Location (2010)
%% \bibitem{ref_url1}
%% LNCS Homepage, \url{http://www.springer.com/lncs}. Last accessed 4
%% Oct 2017
%% \end{thebibliography}
\end{document}
