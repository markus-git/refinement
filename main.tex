% This is samplepaper.tex, a sample chapter demonstrating the
% LLNCS macro package for Springer Computer Science proceedings;
% Version 2.20 of 2017/10/04
%
\documentclass[runningheads]{llncs}
%
\usepackage{xcolor} % For coloring todo-text.
\usepackage{amsmath}
%% \usepackage{amsthm}
\usepackage{amssymb}
\usepackage{stmaryrd} % Double square brackets
\usepackage{graphicx} % Figures
%% \renewcommand\UrlFont{\color{blue}\rmfamily}
%%%%%%%%%%%%%%%%%%%%%%%%%%%%%%%%%%%%%%%%%%%%%%%%%%%%%%%%%%%%%%%%%%%%%%%%%%%%%%%%
% New commands
%%%%%%%%%%%%%%%%%%%%%%%%%%%%%%%%%%%%%%%%%%%%%%%%%%%%%%%%%%%%%%%%%%%%%%%%%%%%%%%%
%
%% Editing.
\newcommand{\todo}[1]{\textcolor{red}{#1}}
\newcommand{\Stmt}[1]{\textrm{\textbf{#1}}}
%% Math
\newcommand{\Tau}{\mathrm{T}}
\newcommand{\ordered}{\sqsubseteq}
\newcommand{\rordered}{\sqsupseteq}
\newcommand{\meet}{\sqcap}
\newcommand{\join}{\sqcup}
\newcommand{\pow}{\wp} % Powerset symbol {mathcal{P}}
\renewcommand{\iff}{\Leftrightarrow} % if and only if
\renewcommand{\implies}{\Rightarrow} % implication
\newcommand{\vsep}{\quad\quad\quad\quad} % Big h. sep.
%% \newcommand{\msim}{\mkern-6mu\sim} % partition sim. used after /.
%% Orig. STE.
\newcommand{\X}{\mathrm{X}} % "Unknown" value.
\DeclareMathOperator{\Next}{\textbf{N}} % Next-time.
\DeclareMathOperator{\Is}{\mathbin{\textbf{is}}} % Comparison.
\DeclareMathOperator{\Traj}{\textit{Traj}} % Set of trajectories
\DeclareMathOperator{\Runs}{\textit{Runs}} % Set of runs
\DeclareMathOperator{\Antecedent}{\pi_{\textrm{a}}}
\DeclareMathOperator{\Consequent}{\pi_{\textrm{c}}}
\newcommand{\aModels}{\models_{\textrm{a}}}
\newcommand{\cModels}{\models_{\textrm{c}}}
%% \newcommand{\sModels}{\models_{\textrm{set}}}
%% STE - set-theoretic.
\newcommand{\sC}{\mathbb{C}}
\newcommand{\sD}{\mathbb{D}}
\newcommand{\RunsFin}[1]{\Runs(#1)^{n}}
\newcommand{\TrajFin}[1]{\Traj(#1)^{n}}
\newcommand{\TrajInf}[1]{\Traj(#1)^{\infty}}
\newcommand{\Refines}{\leq}
\newcommand{\sRefines}{\Refines_{\textrm{set}}}
\newcommand{\sEq}[1]{[#1]}
%% STE - lattice-theoretic simulation.
\newcommand{\lalpha}{\hat{\alpha}}
\newcommand{\lgamma}{\hat{\gamma}}
\newcommand{\lP}{\hat{\mathbb{P}}}
\newcommand{\lQ}{\hat{\mathbb{Q}}}
\DeclareMathOperator{\lAntecedent}{\hat{\pi}_{\textrm{a}}}
\DeclareMathOperator{\lConsequent}{\hat{\pi}_{\textrm{c}}}
\newcommand{\lModels}{\models_{\textrm{lat}}}
\newcommand{\lRefines}{\Refines_{\textrm{lat}}}
\newcommand{\lEq}[1]{\llbracket #1 \rrbracket}
%
%%%%%%%%%%%%%%%%%%%%%%%%%%%%%%%%%%%%%%%%%%%%%%%%%%%%%%%%%%%%%%%%%%%%%%%%%%%%%%%%
% Document
%%%%%%%%%%%%%%%%%%%%%%%%%%%%%%%%%%%%%%%%%%%%%%%%%%%%%%%%%%%%%%%%%%%%%%%%%%%%%%%%

\begin{document}

\title{Refinement for Symbolic Trajectory Evaluation}

\titlerunning{Refinement for STE}

\author{Author \and Author}

\authorrunning{Authors}

\institute{Chalmers}

\maketitle % typeset the header of the contribution

\begin{abstract}
Model refinement such that it preserves symbolic trajectory evalutions.

\keywords{STE \and Refinement \and ?}
\end{abstract}

\section{Symbolic Trajectory Evaluation}

\textit{Symbolic trajectory evaluation}~\cite{seger1995} (STE) is a high-performance model checking technique based on \textit{symbolic simulation} extended with a temporal \textit{next-time} operator to describe circuit behaviour over time. In its simplest form, STE tests the validity of an \textit{assertion} of the form $A \Rightarrow C$, where both the \textit{antecedent} $A$ and \textit{consequent} $C$ are formulas in the following logic:

\begin{equation*}
f ::= p \: | \: f \wedge f \: | \: P \rightarrow f \: | \: \Next f
\end{equation*}

\noindent Here, $p$ is a simple predicate over ``values'' in a circuit and $P$ is a Boolean propositional formula, and the operators $\wedge$, $\rightarrow$ and $\Next$ are conjunction, domain restriction and the next-time operator, respectively.

If the circuit contains Boolean signals, $p$ is typically drawn from the following two predicates: $n \Is 1$ and $n \Is 0$, where $n$ ranges over the signals (or nodes) in a circuit. For example, suppose we have a unit-delayed, two-input AND-gate, then it is reasonable to assume that the assertion $(\mathit{in_{1}} \Is 1 \wedge \mathit{in_{2}} \Is 1) \Rightarrow \Next (\mathit{out} \Is 1)$ is true. Indeed, STE efficiently validates such statements for us.

While the truth semantics of an assertion in STE is defined as the satisfaction of its ``defining'' trajectory (bounded sequence of states) relative to a model structure of the circuit, what the STE algorithm computes is exactly the solution of a data-flow equation~\cite{chou1999} in the classic format~\cite{muchnick1997}. \dots

% translate an STE assertion into a linear, directed graph \dots introducing as many states as the assertion is deep \dots labelling each state with the antecedent and consequent at that depth. \dots

% We adopt this reimagined STE as data flow analysis~\cite{chou1999}, in which trajectory assertions can have arbitrary state-transition graphs. The following sections give a short introduction to its mathematical foundation.

\section{Set-theoretic STE} \label{sec:set-ste}

Consider an arbitrary, but fixed, digital circuit $M$ operating in discrete time. A \textit{configuration} of $M$, denoted by $\sC$, is non-empty and finite set that represents a snapshot of $M$ at a discrete point in time. If the circuit $M$ has $m$ boolean signals, then its set of configurations is typically represented as a sequence $\mathbb{B}^{m}$, where $\mathbb{B} = \{ 0,1 \}$ is the set of boolean values.

\subsubsection{Circuit Model} A simple conceptual model of $M$ is a \textit{transition relation}, $M_{R} \subseteq \sC \times \sC$, where $(c,c') \in M_{R}$ means that $M$ can move from $c$ to $c'$ in one step\footnotemark. The power set of $\sC$, denoted by $\pow(\sC)$, can be viewed as a the set of \textit{predicates} on configurations, where $\cap$, $\cup$, and $\subseteq$ correspond to conjunction, disjunction and implication, respectively. We denote by $\cap S$ and $\cup S$ the intersection and union of all members of any $S \subseteq \pow(\sC)$.

\footnotetext{\todo{Mention how this affects circuits with zero-delays?}}

$M_{R}$ induces a \textit{predicate transformer} $M_{F} \in \pow(\sC) \rightarrow \pow(\sC)$ using the relational image operation:
%
\begin{equation*}
M_{F}(C) = \{ c' \in \sC \mid \exists c \in C : (c,c') \in M_{R} \}
\end{equation*}

\noindent It is intuitively obvious that if $M$ is in one of the configurations in $C \in \pow(\sC)$, then in one time step it must be in one of the configurations in $M_{F}(p)$. We also see that $M_{F}$ distributes over arbitrary unions:
%
\begin{equation*}
M_{F}(\cup S) = \cup \{ M_{F}(C) \mid C \in S \}
\end{equation*}

\noindent for all $S \subseteq \pow(\sC)$. In general, any $M_{F}$ that satisfies this distributive property also defines a $M_{R}$ through the equivalence $(c,c') \in M_{R} \iff c' \in M_{F}(\{ c \})$, that is to say, there is no loss of information going from $M_{R}$ to $M_{F}$ or vice versa. We adopt this functional model of $M$ and drop its subscript.

% It follows that $M_{F}$ preserves the empty set of constraints, i.e. $M_{F}(\emptyset) = \emptyset$, and is monotonic, i.e. $p \subseteq q \implies M_{F}(p) \subseteq M_{F}(q)$ for all $p, q \in \pow(C)$.

Exactly what $\sC$ and its signals are, is not important in this section. In practice, however, signals are typically divided into external, such as ``inputs'' and ``outputs'', and internal parts. While an input signal is generally controlled by the external environment, and thus unconstrained by $M$ itself, non-input signals are determined by the circuit topology and functionality. For example, supposed $M$ is the earlier example of a unit-delayed two-input AND gate, we could then define its model $M \in \pow(\mathbb{B}^{3}) \rightarrow \pow(\mathbb{B}^{3})$ as follows:
%
\begin{equation*}
M(C) = \{ \langle b_{1}, b_{2}, i_{1} \wedge i_{2} \rangle \in \mathbb{B}^{3} \mid \langle i_{1}, i_{2}, o \rangle \in C \}
\end{equation*}

\noindent Here $i_{1}$ and $i_{2}$ refer to the two inputs of the AND gate, $o$ the ignored output, and $b_{1}$ and $b_{2}$ are unconstrained inputs for the new configurations.

\subsubsection{Assertions and Satisfaction} \label{sec:set-ste-sat}

A \textit{trajectory assertion} for $M$ is quintuple $A = (S, s_{0}, R, \Antecedent, \Consequent)$, where $S$ is a finite set of \textit{states}, $s_{0} \in S$ is an \textit{initial state}, $R \subseteq S \times S$ is a \textit{transition relation}, $\Antecedent \in S \rightarrow \pow(\sC)$ and $\Consequent \in S \rightarrow \pow(\sC)$ label each state $s$ with an \textit{antecedent} $\Antecedent(s)$ and a \textit{consequent} $\Consequent(s)$. We assume that $(s,s_{0}) \notin S$ for all $s \in S$ without any loss of generality.

The circuit model $M$ intuitively \textit{satisfies} an assertion $A$ if, for every \textit{trajectory} $\tau$ through $M$ and every \textit{run} $\rho$ through $A$, $\tau$ satisfying the antecedents of $\rho$ entails that $\tau$ also satisfies the consequents of $\rho$. To be specific, a \textit{trajectory} of $M$ is a non-empty sequences of configurations, $\tau \in \sC^{+}$, such that $\tau_{n} \in M(\{ \tau_{n-1} \})$ for all $n \in \mathbb{N} : 0 < n < | \tau |$. And a \textit{run} of $A$ is a non-empty sequence of states, $\rho \in S^{+}$, such that $\rho_{0} = s_{0}$ and $(\rho_{n-1}, \rho_{n}) \in R$ for all $n \in \mathbb{N} : 0 < n < | \rho |$. A finite trajectory $\tau$ satisfies the antecedents of a finite run $\rho$, denoted by $\tau \aModels \rho$, iff $\tau_{n} \in \Antecedent(\rho_{n})$ for all $n \in \mathbb{N} : n < | \tau | = | \rho |$; satisfaction of consequents is defined similarly with $\Consequent$ and denoted by $\tau \cModels \rho$.

That $M$ satisfies $A$, denoted by $M \models A$, can then be formalized as follows: % \footnotemark
%
\begin{equation*}
\forall \tau \in \TrajFin{M} : \forall \rho \in \RunsFin{A} : | \tau | = | \rho | \implies (\tau \aModels \rho \implies \tau \cModels \rho)
\end{equation*}

\noindent where $\TrajFin{M}$ and $\RunsFin{A}$ denote the sets of all finite trajectories of $M$ and runs of $A$, respectively. \todo{This satisfaction can be formulated equivalently as a problem for deterministic finite automaton.}

%% \footnotetext{\todo{This is equivalent to a DFA formulation~\cite{chou1999}.}}

%% For all functions $\Phi \in S \rightarrow \pow(C)$ and states $s \in S$, define $F \in S \rightarrow (\pow(C) \rightarrow \pow(C))$ as follows:

%% \begin{align}
%% F(s)(p) &= M(\Antecedent(s) \cap p) \\
%% \mathcal{F}(\Phi)(s) &= \Stmt{if } (s = s_{0}) \Stmt{ then } C \Stmt{ else } \cup \{ F(s')(\Phi(s')) \mid (s',s) \in R \}
%% \end{align}

%% \noindent $F$ preserves $\emptyset$, and both $F$ and $\mathcal{F}$ are montonic; two $\Phi, \Phi' \in S \rightarrow \pow(C)$ are ordered as $\Phi \subseteq \Phi' \iff \forall s \in S : \Phi(s) \subseteq \Phi'(s)$. Let $\Phi_{*} \in S \rightarrow \pow(C)$ be the least fixpoint of the equation $\Phi = \mathcal{F}(\Phi)$~\cite{davey2002}. Since both $S$ and $\pow(C)$ are finite, $\Phi_{*}$ is given by $\lim \, \Phi_{n}(s)$, where $\Phi_{n}$ is defined as follows:

%% \begin{equation}
%% \Phi_{n} = \Stmt{if } (n = 0) \Stmt{ then } (\lambda s \in S : \bot) \Stmt{ else } \mathcal{F}(\Phi_{n-1})
%% \end{equation}

%% $M$ \textit{satisfies} a trajectory assertion $A$, denoted by $M \smodels A$, iff $\forall s \in S : \Phi_{*}(s) \cap \Antecedent(s) \subseteq \Consequent(s)$.

\section{Refinement}

\subsection{Set-Theoretic refinement}

Consider another fixed, but arbitrary, circuit model $N \in \pow(\D) \rightarrow \pow(\D)$, where $\D$ is a non-empty and finite set of configurations. Further, let there be a Galois connection between predicates $\pow(\C)$ and $\pow(\D)$ ordered by set inclusion and given by means of a binary relation $\ll \, \subseteq \pow(\C) \times \pow(\D)$, where $C \ll D$ reads ``c can be approximated by d''. We define $\ll$ from either of the usual functions for abstraction $\alpha \in \pow(\C) \rightarrow \pow(\D)$ and concretisation $\gamma \in \pow(\D) \rightarrow \pow(\C)$:

\begin{equation*}
C \ll D \iff \alpha(C) \subseteq D \vsep C \ll D \iff C \subseteq \gamma(D)
\end{equation*}

\noindent Here $\subseteq$ on the $\alpha$-derivation side is the inclusion order of $\pow(\D)$, and on the $\gamma$-derivation side $\subseteq$ is the inclusion order of $\pow(\C)$. Intuitively, the relation $\ll$ acts as an extension of the orderings inside $\pow(\C)$ and $\pow(\D)$ to one between them. Note that $\{ \C \}$ and $\{ \D \}$ represent the ``unknown'' states where nothing is assumed, and we thus require that $\ll$ respects this property, i.e. $\{ \C \} = \gamma(\{ \D \})$.

Exactly what configurations such as $\C$ and $\D$ are, were not important previously. To reason about refinement, which relates the external behaviour of circuits, we make a distinction between the \textit{visible} and \textit{internal} elements of $\C$ and $\D$. Let $\sim \, \subseteq \C \times \C$ be an equivalence relation on $\C$; the equivalence class of any $c \in \C$ under $\sim$, denoted by $[c]$, is defined as $[c] = \{ c' \in \C | c' \sim c \}$ and identifies the class of configurations which are \textit{visually equivalent} to $c$. With a slight abuse of notation, we extend $[\cdot]$ to sequences component-wise and overload both $\sim$ and $[\cdot]$ to configurations in $\D$.

We can now formalize that $M$ \textit{refines} $N$, denoted by $M \Refines N$, as follows:

\begin{equation*}
\forall \tau \in \Traj(M) : \exists \upsilon \in \Traj(N) : | \tau | = | \upsilon | \wedge [\tau] \ll [\upsilon]
\end{equation*}

\noindent where $[\tau] \ll [\upsilon]$ iff $[\tau_{n}] \ll [\upsilon_{n}]$ for all $n \in \mathbb{N} : n < | \tau | = | \upsilon |$. In other words, $M \Refines N$ states that, for every sequence of configurations $\tau$ permitted by $M$, there must exist a sequence $\upsilon$ for $N$ which approximates the visual behaviour of $\tau$.

Recall that a trajectory assertion for $N$ is a quintuple $\TA = (S, s_{0}, R, \Antecedent, \Consequent)$, where $\Antecedent \in S \rightarrow \pow(\D)$ and $\Consequent \in S \rightarrow \pow(\D)$ label each $s \in S$ with its antecedents and consequents, respectively. If these antecedents and consequents accept partitions of visible elements in $\D$, i.e. $d \in \Antecedent(s) \iff [d] \subseteq \Antecedent(s)$ and $d \in \Consequent(s) \iff [d] \subseteq \Consequent(s)$ for all $s \in S$, then we refer to $\TA$ as a \textit{external trajectory assertion} and suffix it as $\ETA$. Furthermore, we define $\gamma(\TA) = (S, s_{0}, R, \gamma(\Antecedent), \gamma(\Consequent))$, where $\gamma(\Antecedent) = \lambda s \in S : \gamma(\Antecedent(s))$ and $\gamma(\Consequent) = \lambda s \in S : \gamma(\Consequent(s))$.

%% If the co-domain of $\Antecedent$ and $\Consequent$ is limited to partitions of visible elements in $D$, i.e. $\pow(D / \mkern-8mu\sim)$,

\begin{theorem} \label{thm:traj-refines}
$M \Refines N \implies (N \models \ETA \implies M \models \gamma(\ETA))$
\end{theorem}

The above definition of refinement can be equivalently formulated as a simulation relation. We say that $M$ refines $N$ by \textit{set-theoretic refinement}, denoted by $M \SetRefines N$, iff $\C$ is approximated by $\D$, i.e. $\{ \C \} \ll \{ \D \}$; $\ll$ is a \todo{explanation}, i.e. $C \ll D \implies \bigcup_{c \in C}[c] \ll \bigcup_{d \in D}[d]$; and $\ll$ is a \textit{simulation relation} from $\pow(\C)$ to $\pow(\D)$, i.e. $C \ll D \implies M(C) \ll N(D)$.

%% That $\ll$ is a \todo{explanation} and a simulation relation can also be stated in terms of the usual functions for abstraction $\alpha$ and concretisation $\gamma$.

\begin{theorem} \label{thm:traj-equal-set}
$M \Refines N \iff M \SetRefines N$
\end{theorem}

%\section{Lattice-theoretic STE} \label{sec:lat-ste}

Manipulating subsets of $\mathbb{B}^{m}$ is impractical for even moderately large $m$, which leads us to one of the key insights of STE. Namely, instead of manipulating subsets of $\mathbb{B}^{m}$ directly, one can use sequences of ternary values $\mathbb{T} = \mathbb{B} \cup \{ \X \} $ to approximate them, whose sizes are only linear in $m$. Here the $1$ and $0$ from $\mathbb{B}$ denotes specific, defined values whereas $\X$ denotes an ``unknown'' value that could be either $1$ or $0$. This intuition induces a partial order $\sqsubseteq$ on $\mathbb{T}$, where $0 \sqsubseteq \X$ and $1 \sqsubseteq \X$\footnotemark. For any $m \in \mathbb{N}$, this ordering on $\mathbb{T}$ is lifted component-wise to $\mathbb{T}^{m}$.

% As an example, we have that $\langle 1,1,0 \rangle$ and $\langle 1,0,0 \rangle$ are both $\sqsubseteq \langle 1,X,0 \rangle \in \mathbb{T}^{3}$ because they all agree on their first and third element and $X$ can be both $0$ and $1$.

\footnotetext{\todo{We use the reverse ordering of what is originally used in STE.}}

Note that $\mathbb{T}^{m}$ does not quite form a complete lattice because it lacks a bottom: both $0 \sqsubseteq \X$ and $1 \sqsubseteq \X$ but $0$ and $1$ are equally defined. A special bottom element $\bot$ is therefore introduced, such that $\bot \sqsubseteq t$ and $\bot \neq t$ for all $t \in \mathbb{T}^{m}$. The extended $\mathbb{T}_{\bot}^{m} = \mathbb{T}^{m} \cup \{ \bot \}$ then becomes a complete lattice. We denote the top element $\langle \X, \dots, \X \rangle$ of $\mathbb{T}_{\bot}^{m}$ by $\top$.

\subsubsection{Ternary lattices} \label{sec:lat-ste-intro}

Generalising from any specific domain, let $(\PP, \sqsubseteq)$ be a finite, complete lattice of \textit{abstract predicates} in which the meet $\meet$ and join $\join$ of any subset $\hat S \subseteq \PP$ exists. Similar to the previous set operations for power sets, $\meet$, $\join$ and $\sqsubseteq$ correspond to conjunction, disjunction and implication for abstract predicates, respectively. Furthermore, for any $\hat S \subseteq \PP$, we denote by $\meet \hat S$ and $\join \hat S$ the meet and join of all members of $\hat S$.

Let there be a Galois connection relating ``concrete'' predicates $\pow(\C)$ and abstract predicates $\PP$. The usual definition of a Galois connection is in terms of an \textit{abstraction} $\alpha \in \pow(\C) \rightarrow \PP$ and a \textit{concretisation} $\gamma \in \PP \rightarrow \pow(\C)$ function, such that $\alpha(C) \sqsubseteq \hat p \iff C \subseteq \gamma(\hat p)$ for all $C \in \pow(\C)$ and $\hat p \in \PP$. For example, a Galois connection from $\pow(\mathbb{B}^{m})$ to $\mathbb{T}_{\bot}^{m}$ for any $m \in \mathbb{N}$ can be defined in a natural way by its concretisation function $\gamma \in \mathbb{T}_{\bot}^{m} \rightarrow \pow(\mathbb{B}^{m})$:

\begin{align*}
\gamma ( \langle t_{0}, \dots,t_{m-1} \rangle ) &= \{ \langle b_{0}, \dots,b_{m-1} \rangle \in \mathbb{B}^{m} \mid \forall i < m : t_{i} \neq \X \Rightarrow b_{i} = t_{i} \} \\
\gamma ( \bot ) &= \emptyset
\end{align*}

\noindent Listing each concrete predicate approximated by a given abstract predicate. Its abstraction function $\alpha \in \pow(\mathbb{B}^{m}) \rightarrow \mathbb{T}_{\bot}^{m}$ instead finds the most precise abstract predicate for a set of concrete predicates:

\begin{align*}
\alpha ( C ) &= \join \{ \langle t_{0}, \ldots, t_{m-1} \rangle \in \mathbb{T}_{\bot}^{m} \mid \langle b_{0}, \ldots, b_{m-1} \rangle \in C, \forall i < m : b_{i} = t_{i} \} \\
\alpha ( \emptyset ) &= \bot
\end{align*}

\subsubsection{Abstract circuit model} \label{sec:lat-ste-model}

An \textit{abstract predicate transformer} $\hat M \in \PP \rightarrow \hat P$ is an \textit{abstract interpretation}~\cite{cousot1996} of $M \in \pow(\C) \rightarrow \pow(\C)$ iff: $\hat M$ preserves $\bot$, i.e. $\hat M(\bot) = \bot$; $\hat M$ is monotonic, i.e. $\hat p \sqsubseteq \hat q \Rightarrow \hat M (\hat p) \sqsubseteq \hat M (\hat q)$ for all $\hat p, \hat q \in \PP$; and $\alpha$, or $\gamma$, form a \textit{simulation relation} between $\pow(\C)$ and $\hat P$, i.e. $M(\gamma(\hat p)) \subseteq \gamma(\hat M(\hat p))$ for all $\hat p \in \PP$, or $\alpha(M(C)) \sqsubseteq \hat M(\alpha(C))$ for all $C \in \pow(\C)$.

% \todo{and its Galois connection between $\pow(C)$ and $\hat P$ is a \textit{simulation relation}}

% and $\ll$ is a \textit{simulation relation} from $\pow(C)$ to $\hat P$, i.e. $p \ll \hat p \Rightarrow M (p) \ll \hat M (\hat p)$ for all $p \in \pow(C)$ and $\hat p \in \hat P$.

% That $\ll$ is a simulation relation can also be stated in terms of its abstraction $\alpha$ and concretisation $\gamma$ functions: $\alpha(M(p)) \sqsubseteq \hat M(\alpha(p))$ for all $p \in \pow(C)$, and $M(\gamma(\hat p)) \subseteq \gamma(\hat M(\hat p))$ for all $\hat p \in \hat P$.

Unlike its concrete model, $\hat M$ does not distribute over arbitrary join because information is potentially discarded by the ternary logic durin a join. As an example, let the following $\hat M$ abstract an unit-delayed two-input AND gate:

% ..., that would have been kept by union in the original model $M$

\begin{equation*}
\begin{array}{llll}
  \hat M(\langle 1, 1, \hat p_{2} \rangle) & = \langle \X, \X, 1 \rangle \qquad & \hat M(\langle 0, 0, \hat p_{2} \rangle) & = \langle \X, \X, 0 \rangle \\
  \hat M(\langle 0, \X, \hat p_{2} \rangle) & = \langle \X, \X, 0 \rangle & \hat M(\langle \X, 0, \hat p_{2} \rangle) & = \langle \X, \X, \X \rangle \\
  \hat M(\langle \hat p_{0}, \hat p_{1}, \hat p_{2} \rangle) & = \langle \X, \X, \X \rangle & &
\end{array}
\end{equation*}

% Note: Can we shorten the def. of funny M somehow? Or use a simpler gate.

\noindent where the last, most general matching is overlapped by the more concrete ones. If we apply $\hat M$ to the join of $\langle 0, 1, \X \rangle$ and $\langle 1, 0, \X \rangle$, or if we apply $\hat M$ to them individually and then join, we get two different results:

\begin{equation*}
\begin{array}{lll}
  \hat M(\langle 0, 1, \X \rangle \join \langle 1, 0, \X \rangle) &= \hat M(\langle \X, \X, \X \rangle) &= \langle \X, \X, \X \rangle \\
  \hat M(\langle 0, 1, \X \rangle) \join \hat M(\langle 1, 0, \X \rangle) &= \langle \X, \X, 0 \rangle \join \langle \X, \X, 0 \rangle &= \langle \X, \X, 0 \rangle
\end{array}
\end{equation*}

\noindent The inequality $\join \{ \hat M(\hat p) \mid \hat p \in \hat S \} \sqsubseteq \hat M(\join \hat S)$ for all $\hat S \sqsubseteq \hat P$ does however hold, since it is implied by the monotonicity of $\hat M$.

\subsubsection{Assertions and satisfaction} \label{sec:lat-ste-sat}

A trajectory assertion for an abstract model $\hat M$ is a quintuple $\hat \TA = (S, s_{0}, R, \hat{\pi}_{a}, \hat{\pi}_{c})$, where $S$, $s_{0}$, and $R$ are as in section~\ref{sec:set-ste-sat} and $\hat{\pi}_{a} \in S \rightarrow \PP$ and $\hat{\pi}_{c} \in S \rightarrow \PP$ label each state $s$ with an abstract predicate for its antecedent and consequent, respectively.

\todo{Here follows the definition in~\cite{chou1999}.} For all functions $\hat \Phi \in S \rightarrow \PP$ and states $s \in S$, define $\hat F \in S \rightarrow (\PP \rightarrow \PP)$ and $\hat{\mathcal{F}} \in (S \rightarrow \PP) \rightarrow (S \rightarrow \PP)$ as follows:

\begin{align}
\hat F(s)(\hat p) &= \hat M(\Antecedent(s) \meet \hat p) \\
\hat{\mathcal{F}}(\Phi)(s) &= \Stmt{if } (s = s_{0}) \Stmt{ then } \top \Stmt{ else } \join \{ \hat F(s')(\Phi(s')) \mid (s',s) \in R \}
\end{align}

\noindent $\hat F$ preserves $\bot$, and both $\hat F$ and $\hat{\mathcal{F}}$ are monotonic; two $\hat \Phi, \hat \Phi' \in S \rightarrow \PP$ are ordered as $\hat \Phi \sqsubseteq \hat \Phi' \iff \forall s \in S : \hat \Phi(s) \sqsubseteq \hat \Phi'(s)$. Let $\hat \Phi_{*} \in S \rightarrow \PP$ be the least fixpoint of the equation $\hat \Phi = \hat{\mathcal{F}}(\hat \Phi)$~\cite{davey2002}. Since both $S$ and $\PP$ are finite, $\hat \Phi_{*}$ is given by $\lim \, \hat \Phi_{n}(s)$, where $\hat \Phi_{n}$ is defined as follows:

\begin{equation}
\hat \Phi_{n} = \Stmt{if } (n = 0) \Stmt{ then } (\lambda s \in S : \bot) \Stmt{ else } \hat{\mathcal{F}}(\hat \Phi_{n-1})
\end{equation}

We say that $\hat M$ \textit{satisfies} a trajectory assertion\footnotemark $\hat \TA$, denoted by $\hat M \LatModels \hat \TA$, iff $\hat \Phi_{*}(s) \meet \pi_{\alpha}(s) \sqsubseteq \Consequent(s)$ for all $s \in S$.

% That is, $\hat M$ satisfies $\hat A$ if, for every state $s$ in the assertion, the information gathered from $\hat M$ through the consequent in every path to $s$ implies the antecedent for $s$.

\footnotetext{That $\hat M$ satisfies $\hat \TA$ implies that a concretisation of $\hat \TA$ can also be satisfied by the original, set-based model $M$ \cite{chou1999}.}

\subsubsection{Refinement}

\todo{Lattice-theoretic refinement should go here.}

%\subsection{Refinement}

Let the abstract predicate transformer $\hat N \in \lQ \rightarrow \lQ$ be an abstract interpretation of the earlier circuit model $N$, where $\lQ$ is an abstract predicate for which there exists a Galois connection to $\pow(\sD)$.

Let equivalent predicates in $\lP$ be identified by a function $\lEq{\cdot} \in \lP \rightarrow \lP$, such that $\lEq{\cdot}$ (?) preserves top and bottom, i.e. $\lEq{\top} = \top$ and $\lEq{\bot} = \bot$; (?) is idempotent, i.e. $\lEq{(\lEq{\hat p})} = \lEq{\hat p}$; (?) is monotonic, i.e. $\hat p \ordered \hat q \implies \lEq{\hat p} \ordered \lEq{\hat q}$; (?) \todo{increasing}, i.e. $\hat p \ordered \lEq{\hat p}$; (?) \todo{name}, i.e. $\hat q = \lEq{\hat q} \implies \lEq{\hat p \meet \hat q} \iff \lEq{\hat p} \meet \hat q$

Let there exist a Galois connection between $\lP$ and $\lQ$, given by the usual functions for abstraction $\hat \alpha \in \lP \rightarrow \lQ$ and concretisation $\hat \gamma \in \lQ \rightarrow \lP$, such that $\hat \alpha(\hat p) \ordered \hat q \iff \hat p \subseteq \hat \gamma(\hat q)$ for all $\hat p \in \lP$ and $\hat q \in \lQ$.

Let the binary relation $\lll \, \subseteq \lP \times \lQ$ be derived from the above $\hat \alpha$ or $\hat \gamma$:

\begin{equation*}
\hat p \lll \hat q \iff \hat \alpha(\lEq{\hat p}) \ordered \lEq{\hat q} \vsep \hat p \lll \hat q \iff \lEq{\hat p} \ordered \hat \gamma(\lEq{\hat q})
\end{equation*}

\noindent Here $\ordered$ on $\hat \alpha$-derivation side is the partial order of $\lQ$, and on the $\hat \gamma$-derivation side $\ordered$ is the partial order of $\lP$.

Finally, we say that $\hat M$ refines $\hat N$ by \textit{lattice-theoretic simulation}, denoted by $\hat M \lRefines \hat N$, iff (1) $\lll$ is a simulation relation, i.e. $\hat p \lll \hat q \implies \hat M(\hat p) \lll \hat N(\hat q)$. That $\lll$ is a simulation relation can also be stated directly in terms of the usual functions for abstraction, $\hat \alpha(\lEq{\hat M(\hat p)}) \ordered \lEq{\hat N(\hat \alpha(\hat p))}$ for all $\hat p \in \lP$, or concretisation, $\lEq{\hat M(\hat \gamma(\hat q))} \ordered \hat \gamma(\lEq{\hat N(\hat q)})$ for all $\hat q \in \lQ$.

\begin{theorem} \label{thm:lat-refines}
$\hat M \lRefines \hat N \implies (\hat N \lModels \hat{A}_{\todo{n}} \implies \hat M \lModels \hat \gamma(\hat{A}_{\todo{n}}))$
\end{theorem}

\begin{theorem} \label{thm:lat-imply-set}
$\hat M \lRefines \hat N \implies M \sRefines N$
\end{theorem}


\appendix
\section{Appendices}

\subsection{Theorem~\ref{thm:traj-refines}}

We use freely the fact that $[\{ c \}] = [c]$ and first prove a few lemmas.

\begin{lemma} \label{lem:class-sub}
$[d] \cap \Antecedent(\rho) \neq \emptyset \implies [d] \subseteq \Antecedent(\rho)$
\end{lemma}

Since they intersect, there must exist $d' \in [d]$ such that $d' \in \Antecedent(\rho)$. By the invariance of $\Antecedent(\rho)$, it must be that $[d'] = [d] \subseteq \Antecedent(\rho)$. $\qedbox$

\begin{lemma} \label{lem:traj-con}
$d \in \Consequent(\rho) \wedge \{ c \} \ll \{ d \} \implies c \in \gamma(\Consequent(\rho))$
\end{lemma}

By the invariance of $\Consequent$, $d \in \Consequent(\rho)$ implies that $[d] \subseteq \Consequent(\rho)$, which in turn implies $\gamma([d]) \subseteq \gamma(\Consequent(\rho))$ by the monotonicity of $\gamma$. By definition of $\{ c \} \ll \{ d \}$, we know $[c] \subseteq \gamma([d])$, and thus $[c] \subseteq \gamma(\Consequent(\rho))$. That $c \in \gamma(\Consequent(\rho))$ then follows. $\qedbox$

\begin{lemma} \label{lem:traj-ant}
$c \in \gamma(\Antecedent(\rho)) \wedge \{ c \} \ll \{ d \} \implies d \in \Antecedent(\rho)$
\end{lemma}

By the invariance of $\Antecedent$, and the preservation of it by $\gamma$, $c \in \gamma(\Antecedent(\rho))$ implies that $[c] \subseteq \gamma(\Antecedent(\rho))$. And thus $\alpha([c]) \subseteq \alpha(\gamma(\Antecedent(\rho)) \subseteq \Antecedent(\rho)$ by the monotonicity of $\alpha$. By definition of $\{ c \} \ll \{ d \}$, we also have that $\alpha([c]) \subseteq [d]$. Since $\alpha([c]) \neq \emptyset$, it must be that $[d] \cap \Antecedent(\rho) \neq \emptyset$, and thus $[d] \subseteq \Antecedent(\rho)$ by lemma~\ref{lem:class-sub}. That $d \in \Antecedent(\rho)$ then follows immediately. $\qedbox$
\\

For the theorem, we are given $\tau \in \Traj(M)$ and $\rho \in \Runs(\gamma(A))$, such that $| \tau | = | \rho |$ and $\tau_{n} \in \gamma(\Antecedent(\rho_{n}))$ for all $n \in \mathbb{N} : n < | \tau |$. We must then show that $\tau_{n} \in \gamma(\Consequent(\rho_{n}))$. By the refinement assumption, there must exist a $\upsilon \in \Traj(N)$ such that $| \tau | = | \upsilon | = | \rho |$ and $\{ \tau_{n} \} \ll \{ \upsilon_{n} \}$. Lemma~\ref{lem:traj-ant} then shows that $\upsilon_{n} \in \Antecedent(\rho_{n})$ and, by the assumption, we have $\upsilon_{n} \in \Consequent(\rho_{n})$. Lemma~\ref{lem:traj-con} then shows that $\tau_{n} \in \gamma(\Consequent(\rho_{n}))$. $\qedbox$

\subsection{Theorem~\ref{thm:traj-equal-set}}

We first show a lemma.

\begin{lemma} \label{lem:ll-sub}
$C \ll D \wedge D \subseteq D' \implies C \ll D'$
\end{lemma}

That $C$ is approximated by $D'$ follows immediately: $\alpha(C) \subseteq D \subseteq D'$. The first property of $\ll$ follows from the definition of subset, and the second by the monotonicity of $N$: $\alpha(M(C)) \subseteq N(D) \subseteq N(D')$. $\qedbox$
\\

We prove each direction of the theorem separately.

$(\Rightarrow)$ : If $C \ll D$, then by definition $\alpha([C]) \subseteq [D]$. As $\alpha$ distributes over arbitrary union, it follows that $\alpha([c]) \subseteq [D]$ for all $c \in C$. We note that every such $c \in C$ is also the start of some trajectories in $M$, and it therefore follows from the refinement assumption that there exist a trajectory in $N$ with a start $d \in \sD$ such that $\{ c \} \ll \{ d \}$, or $\alpha([c]) \subseteq [d]$. By the requirement that $\alpha([c]) \neq \emptyset$, it must be that $[d] \cap [D] \neq \emptyset$. By lemma~\ref{lem:class-sub} then, we know $[d] \subseteq [D]$ and thus $d \in D$, which is the first property required of $\ll$. For the second property, that $\ll$ is a simulation relation, consider any ``next-step'' of these trajectories starting in $c$ and $d$, i.e. $c' \in M(\{ c \})$ and $d' \in N(\{ d \})$. From the refinement assumption we know that $\{ c' \} \ll \{ d' \}$, or $\alpha([c']) \subseteq [d'] \subseteq [N(\{ d \})] \subseteq [N(D)]$. Taking the union of every such ordering for $c' \in M(\{ c \})$, we see that $M(\{ c \}) \ll N(D)$ for all $c \in C$, or $M(C) \ll N(D)$, as required.

% there exists a $d' \in N(D)$ such that $\{ c' \} \ll \{ d' \}$, which implies that $\{ c' \} \ll N(D)$ by lemma~\ref{lem:ll-sub}. Combining all such orderings, we have the desired $M(C) \ll N(D)$. $\qedbox$

% i.e. $\alpha([C]) = \cup \{ \alpha([c]) \in \pow(\C) \mid c \in C \}$

$(\Leftarrow)$ : We show this claim by induction on the length of $\tau$. For the base case, $| \tau | = 1$, we are given $\tau = \langle \tau_{1} \rangle$ where $\tau_{1}$ is unconstrained, i.e. we only know that $\tau_{1} \in \sC$. But a Galois connection always relates the most general states of its two partially ordered sets, so $\alpha(\{ \sC \}) \subseteq \{ \sD \}$. As $[\{ \sC \}] = \{ \sC \}$ and $[\{ \sD \}] = \{ \sD \}$, we also have $\alpha([\{ \sC \}]) \subseteq [\{ \sD \}]$, or $\{ \sC \} \ll \{ \sD \}$. Using the first property of $\ll$ then tells us that there exists $d \in \sD$ such that $|\langle \tau_{1} \rangle| = |\langle d \rangle|$ and $\{ \tau_{1} \} \ll \{ d \}$. For the inductive step, $| \tau | = n + 1$, we are given a sequence $\langle \dots, \tau_{n}, \tau_{n+1} \rangle$ and assume there exists another sequence $\langle \dots, \upsilon_{n} \rangle$ such that $|\langle \dots, \tau_{n} \rangle| = |\langle \dots, \upsilon_{n} \rangle|$ and $\langle \dots, \tau_{n} \rangle \ll \langle \dots, \upsilon_{n} \rangle$. From the simulation property of $\ll$, we know that $M(\tau_{n}) \ll N(\upsilon_{n})$ and, by the definition of trajectories, that $\tau_{n+1} \in M(\tau_{n})$. Applying the first property of $\ll$ then states that there exists $d \in N(\upsilon_{n})$ such that $\{ \tau_{n+1} \} \ll \{ d \}$. The concatenation of $\langle \dots, \upsilon_{n+1} \rangle$ and $\langle d \rangle$, i.e. $\langle \dots, \upsilon_{n}, d \rangle$, forms a valid trajectory in $\Traj(N)$ and satisfies the properties $| \langle \dots, \tau_{n}, \tau_{n+1} \rangle | = | \langle \dots, \upsilon_{n}, d \rangle |$ and $\langle \dots, \tau_{n}, \tau_{n+1} \rangle \ll \langle \dots, \upsilon_{n}, d \rangle$. $\qedbox$

% We claim that, for all $\tau \in \Traj(M)$, there exists a $\upsilon \in \Traj(N)$, such that $| \tau | = | \upsilon |$ and $\tau \ll \upsilon$.

\subsection{Theorem~\ref{thm:lat-refines}}

First a lemma that shows $\lEq{\hat p \meet \lAntecedent(s)} \iff \lEq{\hat p} \meet \lAntecedent(s)$ is a reasonable assumption.

\begin{lemma}
$\sEq{D} = D \implies (\sEq{C \cap D} \iff \sEq{C} \cap D)$
\end{lemma}

That $x \in \sEq{C \cap D}$ can be stated as $\exists y : x \sim y \wedge y \in C \wedge y \in D$. That $y \in D$ implies that $x \in [D] = D$ since $x \sim y$, and we can therefore restate $x \in \sEq{C \cap D}$ as $\exists y : x \sim y \wedge y \in C \wedge x \in D \iff x \in \sEq{C} \wedge x \in D \iff x \in \sEq{C} \cap D$. $\qedbox$
\\

%% \begin{align*}
%% x \in \sEq{C \cap D} & \iff \exists y : x \sim y \wedge y \in C \wedge y \in D \\
%%                      & \iff \exists y : x \sim y \wedge y \in C \wedge x \in D \\
%%                      & \iff x \in \sEq{C} \wedge x \in D \\
%%                      & \iff x \in \sEq{C} \cap D
%% \end{align*}

Secondly, we show a few helpful lemmas that regard the fix-points and functions used to determine satisfaction with $\hat M$ and $\hat N$. To clarify which definitions are used with $\hat N$ and which are used with $\hat N$, we will duplicate some earlier definitions. Specifically, let $G$, $\mathcal{G}$ and $\Psi$ be equivalent operations for $\hat N$, as $F$, $\mathcal{F}$ and $\Phi$ are for $\hat M$:

\begin{align*}
G(s)(\hat q) &= \hat N(\lalpha(\pi_{a}(s)) \sqcap \hat q) \\
\mathcal{G}(\Psi)(s) &= \Stmt{if } (s = s_{0}) \Stmt{ then } \top \Stmt{ else } \sqcup \{ G(s')(\Psi(s')) \mid (s',s) \in R \} \\
\Psi_{n} &= \Stmt{if } (n = 0) \Stmt{ then } (\lambda s \in S : \bot) \Stmt{ else } \mathcal{G}(\Psi_{n-1}) \\
\end{align*}

\noindent Furthermore, let $\Psi_{*}$ be the least fixpoint of $\Psi = \mathcal{G}(\Psi)$ and given by $\lim \, \Psi_{n}(s)$.

\begin{lemma} \label{lem:bot-refine-bot}
$\bot \lll \bot$ \& $\top \lll \top$
\end{lemma}

A Galois connection always relates the two tops and bottoms, i.e. $\lalpha(\top) \ordered \top$ and $\lalpha(\bot) \ordered \bot$. Because $\lEq{\cdot}$ preserves both tops and bottoms, it follows that both $\lalpha(\lEq{\top}) \ordered \lEq{\top}$, or $\top \lll \top$, and $\lalpha(\lEq{\bot}) \ordered \lEq{\bot}$, or $\bot \lll \bot$, holds as well.

%% \begin{lemma} \label{lem:closed-join}
%% $\hat p \lll \hat q \wedge \hat r \lll \hat s \implies (\hat p \join \hat r) \lll (\hat q \join \hat s)$
%% \end{lemma}

%% Expanding $\lll$, we have $\hat \alpha(\lEq{\hat p}) \ordered \lEq{\hat q} \ordered \lEq{\hat q \join \hat s}$ and $\hat \alpha(\lEq{\hat r}) \ordered \lEq{\hat s} \ordered \lEq{\hat q \join \hat s}$ by the monotonicity of $\lEq{\cdot}$. Hence $\hat \alpha(\lEq{\hat p}) \join \hat \alpha(\lEq{\hat r}) = \hat \alpha(\lEq{\hat p \join \hat r}) \ordered \lEq{\hat q \join \hat s}$ as $\lEq{\cdot}$ and $\hat \alpha$ distribute over arbitrary union. \todo{Similarly}, $\hat \alpha(\hat M(\hat p)) \ordered \hat N(\hat q) \ordered \hat N(\hat q \join \hat s)$ and $\hat \alpha(\hat M(\hat r)) \ordered \hat N(\hat s) \ordered \hat N(\hat q \join \hat s)$. \todo{Thus} $\hat \alpha(\hat M(\hat p)) \join \hat \alpha(\hat M(\hat r)) = \hat \alpha(\hat M(\hat p) \join \hat M(\hat r)) \ordered \hat N(\hat q \join \hat s)$.

\begin{lemma} \label{lem:1}
$\hat p \lll \hat q \implies \forall s \in S : F(s)(\hat p) \lll G(s)(\hat q)$
\end{lemma}

\todo{Is the required step $\lgamma(\hat q) \rordered \hat p$ implied by $\hat p \lll \hat q$ or no?}

\begin{align*}
\lgamma(\lEq{G(s)(\hat q)})
  & =         \lgamma(\lEq{\hat N(\lAntecedent(s) \meet \hat q)}) \\
  & \rordered \lEq{\hat M(\lgamma(\lAntecedent(s) \meet \hat q))} \\
  & =         \lEq{\hat M(\lgamma(\lAntecedent(s)) \meet \lgamma(\hat q)))} \\
  & \rordered \lEq{\hat M(\lgamma(\lAntecedent(s)) \meet \hat p))} \\
  & =         \lEq{F(s)(\hat p)}
\end{align*}

\begin{lemma} \label{lem:3}
$\dots \implies \forall s \in S : \Phi_{*}(s) \lll \Psi_{*}(s)$
\end{lemma}

Since $\Phi_{*}(s) = \lim \, \Phi_{n}(s)$ and $\Psi_{*}(s) = \lim \, \Psi_{n}(s)$, it suffices to prove that $\Phi_{n}(s) \lll \Psi_{n}(s)$ for all $s \in S$ and $n \in \mathbb{N}$. We do so by induction on $n$. The base case, where $\Phi_{0}(s) = \bot$ and $\Psi_{0}(s) = \bot$, follows from lemma~\ref{lem:bot-refine-bot}. For the inductive step, assume that $\Phi_{n}(s) \lll \Psi_{n}(s)$ for all $s \in S$. For $s = s_{0}$, we have that $\Phi_{n+1}(s_{0}) = \top$ and $\Psi_{n+1}(s_{0}) = \top$, which also follows from lemma~\ref{lem:bot-refine-bot}. For any $s \neq s_{0}$, we have:

\begin{align*}
\lgamma(\lEq{\Psi_{n+1}})
  & =         \lgamma(\lEq{\mathcal{F}(\Psi_{n})(s)}) \\
  & =         \lgamma(\lEq{\join \{ F(s')(\Psi_{n}(s')) \mid (s',s) \in R \}}) \\
  & \rordered \dots
\end{align*}
\\

Consider an arbitrary $s \in S$ and let $\hat p = \Phi_{*}(s) \meet \lgamma(\lAntecedent(s))$, where $\hat p \ordered \Phi_{*}(s)$ and $\hat p \ordered \lgamma(\lAntecedent(s))$, or $\lalpha(\hat p) \ordered \lAntecedent(s)$. From lemma~\ref{lem:3} we know $\Phi_{*}(s) \lll \Psi_{*}(s)$, and thus $\lalpha(\hat p) \ordered \lalpha(\lEq{\hat p}) \ordered \lalpha(\lEq{\Phi_{*}(s)}) \ordered \lEq{\Psi_{*}(s)}$ by the monotonicity of $\lEq{\cdot}$ and $\lalpha$. Using the invariance of $\lAntecedent(s)$ and property \todo{name} of $\lEq{\cdot}$, we note that the assumption can be restated as $\lConsequent(s) \rordered \Psi_{*}(s) \meet \lAntecedent(s) = \lEq{\Psi_{*}(s) \meet \lAntecedent(s)} = \lEq{\Psi_{*}(s)} \meet \lAntecedent(s)$. It then follows that $\alpha(\hat p) \ordered \lConsequent(s)$, or $\hat p \ordered \lgamma(\lConsequent(s))$ as desired. $\qedbox$

\subsection{Theorem~\ref{thm:lat-imply-set}}

Text.


\bibliographystyle{splncs04}

\bibliography{lib}

\end{document}
